\documentclass[notheorems]{beamer}
\usepackage{etex}
\usetheme{Madrid}
\usepackage[utf8]{inputenc}
\usepackage[T1]{fontenc}
\usepackage[slovak]{babel}
\usepackage{graphicx}
\usepackage{cite}
\usepackage{url}
\usepackage{amsthm}
\usepackage{amsmath}
\usepackage{mathrsfs}
\usepackage{amssymb}
\usepackage{pifont}% http://ctan.org/pkg/pifont
\usepackage{enumitem}
\setlist[itemize,1]{label={\fontfamily{cmr}\fontencoding{T1}\selectfont\textbullet}}
\usepackage{tikz}
\usepackage{float}
\usepackage{verbatim}
\usepackage{hhline}
\usepackage{ctable}
\usepackage{array}
\usetikzlibrary{automata,arrows,topaths}

\newtheorem{theorem}{Veta}
\newtheorem{definition}[theorem]{Definícia}

\newcommand{\xmark}{\ding{55}}

\newcolumntype{?}{!{\vrule width 3pt}}

\makeatletter
\let\insertsupervisor\relax
\newcommand\supervisortitle{Školiteľ}
\mode<all>
{
  \newcommand\supervisor[1]{\def\insertsupervisor{#1}}
  \titlegraphic{}
}
\defbeamertemplate*{title page}{supdefault}[1][]
{
  \vbox{}
  \vfill
  \begingroup
    \centering
    \begin{beamercolorbox}[sep=8pt,center,#1]{title}
      \usebeamerfont{title}\inserttitle\par%
      \ifx\insertsubtitle\@empty\relax%
      \else%
        \vskip0.25em%
        {\usebeamerfont{subtitle}\usebeamercolor[fg]{subtitle}\insertsubtitle\par}%
      \fi%     
    \end{beamercolorbox}%
    \vskip1em\par
    \begin{beamercolorbox}[sep=8pt,center,#1]{author}
      \usebeamerfont{author}\insertauthor
    \end{beamercolorbox}
    \ifx\insertsupervisor\relax\relax\else
    \begin{beamercolorbox}[sep=8pt,center,#1]{author}
      \usebeamerfont{author}\supervisortitle:~\insertsupervisor
    \end{beamercolorbox}\fi
    \begin{beamercolorbox}[sep=8pt,center,#1]{institute}
      \usebeamerfont{institute}\insertinstitute
    \end{beamercolorbox}
    \begin{beamercolorbox}[sep=8pt,center,#1]{date}
      \usebeamerfont{date}\insertdate
    \end{beamercolorbox}\vskip0.5em
    {\usebeamercolor[fg]{titlegraphic}\inserttitlegraphic\par}
  \endgroup
  \vfill
}
\setbeamertemplate{title page}[supdefault][colsep=-4bp,rounded=true,shadow=\beamer@themerounded@shadow]\makeatother

\title[Prídavná informácia a NKA]{Prídavná informácia a zložitosť nedeterministických konečných automatov}
\subtitle{diplomová práca}
\author{Šimon Sádovský}
\supervisor{Branislav Rovan}
\institute{FMFI UK}
\date{\today}

\begin{document}

\begin{frame}
\maketitle
\end{frame}

\begin{frame}
\frametitle{Úvod do problematiky, motivácia}
\begin{itemize}
\item Chceme nedeterministickým konečným automatom akceptovať jazyk $ \lbrace w \in \lbrace a \rbrace^* \; | \; |w| \equiv 0 \; (mod \; 6) \rbrace $. Koľko stavov potrebujeme?
\end{itemize}

\begin{figure}[H]
\centering
\begin{tikzpicture}[->,>=stealth',shorten >=1pt,auto,node distance=1.7cm,
                    semithick]
   \node[state,initial,accepting] 	(0)					{$q_{0}$}; 
   \node[state] 			(1)		[right of=0] 		{$q_{1}$}; 
   \node[state] 			(2) 	[right of=1] 		{$q_{2}$};
   \node[state] 			(3) 	[right of=2] 		{$q_{3}$};
   \node[state] 			(4) 	[right of=3] 		{$q_{4}$};
   \node[state] 			(5) 	[right of=4] 		{$q_{5}$};
   
   \path[->] 
    (0) edge node {a} (1)
    (1) edge node {a} (2)
    (2) edge node {a} (3)
    (3) edge node {a} (4)
    (4) edge node {a} (5)
    (5) edge [bend right] node [swap] {a} (0)
    ;
\end{tikzpicture}
\end{figure}

\end{frame}


\begin{frame}
\frametitle{Úvod do problematiky, motivácia}

\begin{itemize}
\item Čo ak by sme automatu niečo o vstupe \glqq{} našepkali \grqq{}?
\end{itemize}

\begin{figure}[H]
\centering
\resizebox {\columnwidth} {!} {
	\includegraphics[width=2cm]{whisper-001}
	\begin{tikzpicture}[->,>=stealth',shorten >=1pt,auto,node distance=1.7cm,
    	                semithick]
	   \node[state,initial,accepting] 	(0)					{$q_{0}$}; 
	   \node[state] 			(1)		[right of=0] 		{$q_{1}$}; 
	   \node[state] 			(2) 	[right of=1] 		{$q_{2}$};
	   \node[state] 			(3) 	[right of=2] 		{$q_{3}$};
	   \node[state] 			(4) 	[right of=3] 		{$q_{4}$};
	   \node[state] 			(5) 	[right of=4] 		{$q_{5}$};
   
	   \path[->] 
    	(0) edge node {a} (1)
	    (1) edge node {a} (2)
    	(2) edge node {a} (3)
	    (3) edge node {a} (4)
    	(4) edge node {a} (5)
	    (5) edge [bend right] node [swap] {a} (0)
    ;
	\end{tikzpicture}
}
\end{figure}

\begin{itemize}
\item Ak budeme šepkať, či je dĺžka slova na vstupe delitelná tromi, tak stačí NKA s dvomi stavmi.
\end{itemize}

\begin{figure}[H]
\centering
	\includegraphics[width=2cm]{whisper-001}
	\begin{tikzpicture}[->,>=stealth',shorten >=1pt,auto,node distance=2.2cm,
    	                semithick]
	   \node[state,initial,accepting] 	(0)					{$q_{0}$}; 
	   \node[state] 			(1)		[right of=0] 		{$q_{1}$}; 
   
	   \path[->] 
    	(0) edge [bend right]node [swap] {a} (1)
	    (1) edge [bend right] node [swap] {a} (0)
    ;
	\end{tikzpicture}
\end{figure}

\end{frame}


\begin{frame}
\frametitle{Úvod do problematiky, motivácia}

\begin{itemize}
\item Ak budeme šepkať, či je slovo dĺžky aspoň 78, tak sme automatu vo všeobecnosti velmi nepomohli.
\end{itemize}

\begin{figure}[H]
\centering
\resizebox {\columnwidth} {!} {
	\includegraphics[width=2cm]{whisper-001}
	\begin{tikzpicture}[->,>=stealth',shorten >=1pt,auto,node distance=1.7cm,
    	                semithick]
	   \node[state,initial,accepting] 	(0)					{$q_{0}$}; 
	   \node[state] 			(1)		[right of=0] 		{$q_{1}$}; 
	   \node[state] 			(2) 	[right of=1] 		{$q_{2}$};
	   \node[state] 			(3) 	[right of=2] 		{$q_{3}$};
	   \node[state] 			(4) 	[right of=3] 		{$q_{4}$};
	   \node[state] 			(5) 	[right of=4] 		{$q_{5}$};
   
	   \path[->] 
    	(0) edge node {a} (1)
	    (1) edge node {a} (2)
    	(2) edge node {a} (3)
	    (3) edge node {a} (4)
    	(4) edge node {a} (5)
	    (5) edge [bend right] node [swap] {a} (0)
    ;
	\end{tikzpicture}
}
\end{figure}

\begin{itemize}
\item Skúmame otázku, aké našepkávanie je zmysluplné a pomôže a aké nie.
\item Ako formalizovať tento problém?
\end{itemize}

\end{frame}


\begin{frame}
\frametitle{Definícia problému}

\begin{definition}
\textbf{Stavovou zložitosťou} nedeterministického konečného automatu $ A $ (označujeme \em{}$ \#_S(A) $\em{}) rozumieme počet jeho stavov.
\end{definition}

\begin{definition}
\textbf{Nedeterministickú stavovú zložitosť} jazyka $ L \in \mathscr{R} $ (označujeme \em{}$ nsc(L) $\em{} - z anglického nondeterministic state complexity) definujeme \em{}$ nsc(L)=min \lbrace \#_S(A) | L(A)=L \rbrace $\em{}.
\end{definition}

\begin{definition}
Nech $ L \in \mathscr{R} $. \textbf{Minimálnym nedeterministickým konečným automatom pre jazyk L} rozumieme ľubovolný nedeterministický konečný automat $ A $ taký, že $ \#_S(A)=nsc(L) $.
\end{definition}

\end{frame}


\begin{frame}
\frametitle{Definícia problému}

\begin{definition}
Nech $ A $ je nedeterministický konečný automat. Potom dva nedeterministické konečné automaty $ A_1, A_2 $ také, že $ L(A)=L(A_1) \cap L(A_2) $ nazveme \textbf{rozklad automatu} $ A $. Ak navyše platí $ \#_S(A_1) < \#_S(A)$ a $ \#_S(A_2) < \#_S(A) $, nazývame tento rozklad \textbf{netriviálny}. Ak existuje netriviálny rozklad automatu $ A $, tak automat $ A $ nazývame \textbf{rozložitelný}.
\end{definition}

\begin{definition}
\label{def:nedeterministic_decomposability_of_language}
Nech $ L \in \mathscr{R} $ a $ A $ je nejaký minimálny NKA pre jazyk L. \textbf{Jazyk} $ L $ nazývame \textbf{nedeterministicky rozložitelný} práve vtedy, keď je automat $ A $ rozložitelný.
\end{definition}

\end{frame}


\begin{frame}
\frametitle{Príklad rozložiteľného}
\begin{theorem}
Jazyk $ \lbrace w \in \lbrace a \rbrace^* \; | \; |w| \equiv 0 \; (mod \; 6) \rbrace $ je rozložiteľný.
\end{theorem}

\begin{proof}
\begin{figure}[H]
\centering
\begin{tikzpicture}[->,>=stealth',shorten >=1pt,auto,node distance=1.7cm,
                    semithick]
   \node[state,initial,accepting] 	(0)					{$q_{0}$}; 
   \node[state] 			(1)		[right of=0] 		{$q_{1}$}; 
   \node[state] 			(2) 	[right of=1] 		{$q_{2}$};
   \node[state] 			(3) 	[right of=2] 		{$q_{3}$};
   \node[state] 			(4) 	[right of=3] 		{$q_{4}$};
   \node[state] 			(5) 	[right of=4] 		{$q_{5}$};
   
   \path[->] 
    (0) edge node {a} (1)
    (1) edge node {a} (2)
    (2) edge node {a} (3)
    (3) edge node {a} (4)
    (4) edge node {a} (5)
    (5) edge [bend right] node [swap] {a} (0)
    ;
\end{tikzpicture}
\end{figure}
\begin{figure}[H]
\centering
\resizebox {\columnwidth} {!} {
	\begin{tikzpicture}[->,>=stealth',shorten >=1pt,auto,node distance=2.2cm,
    	                semithick]
	   \node[state,initial,accepting] 	(0)					{$q_{0}$}; 
	   \node[state] 			(1)		[right of=0] 		{$q_{1}$}; 
   
	   \path[->] 
    	(0) edge [bend right]node [swap] {a} (1)
	    (1) edge [bend right] node [swap] {a} (0)
    ;
	\end{tikzpicture}
	
	\begin{tikzpicture}[->,>=stealth',shorten >=1pt,auto,node distance=2.2cm,
    	                semithick]
	   \node[state,initial,accepting] 	(0)					{$q_{0}$}; 
	   \node[state] 			(1)		[right of=0] 		{$q_{1}$};
	   \node[state] 			(2)		[right of=1] 		{$q_{2}$};
   
	   \path[->] 
    	(0) edge node [swap] {a} (1)
    	(1) edge node [swap] {a} (2)
	    (2) edge [bend right] node [swap] {a} (0)
    ;
	\end{tikzpicture}
}
\end{figure}

\end{proof}

\end{frame}


\begin{frame}
\frametitle{Príklad nerozložiteľného}

\begin{theorem}
Jazyk $\lbrace a^5 \rbrace$ je nerozložieľný.
\end{theorem}

\begin{proof}

\begin{figure}[H]
\centering
\begin{tikzpicture}[->,>=stealth',shorten >=1pt,auto,node distance=1.7cm,
                    semithick]
   \node[state,initial] 	(0)					{$q_{0}$}; 
   \node[state] 			(1)		[right of=0] 		{$q_{1}$}; 
   \node[state] 			(2) 	[right of=1] 		{$q_{2}$};
   \node[state] 			(3) 	[right of=2] 		{$q_{3}$};
   \node[state] 			(4) 	[right of=3] 		{$q_{4}$};
   \node[state,accepting] 			(5) 	[right of=4] 		{$q_{5}$};
   
   \path[->] 
    (0) edge [bend left] node {a} (1)
    (1) edge [bend left] node {a} (2)
    (2) edge [bend left] node {a} (3)
    (3) edge [bend left] node {a} (4)
    (4) edge [bend left] node {a} (5)
    ;
\end{tikzpicture}
\end{figure}

\begin{itemize}
\item Nech existuje rozklad, tj. NKA $ A_1, A_2 $ také, že $ L(A_1) \cap L(A_2) = \lbrace a^5 \rbrace, \#_S(A_1) < \#_S(A), \#_S(A_1) < \#_S(A)$.
\item $ a^5 \in L(A_1), \; a^5 \in L(A_2) $
\item lebo málo stavov, tak viem pumpovať nejakú časť $ a^5 $ v $ A_1 $ aj $ A_2 $, t.j. $ \exists k,l \leq 5 \; \forall n: a^{5+kn} \in L(A_1),a^{5+ln} \in L(A_2) $
\item $ a^{5+kl} \in L(A_1) \cap L(A_2) $, spor
\end{itemize}

\end{proof}

\end{frame}


\begin{frame}
	\frametitle{Jazyky založené na dĺžke slov}
	
	\begin{theorem}
		Nech pre $ Z \in \mathbb{N}, Z > 0 $ je $ L_Z = \lbrace a^{kZ} \; | \; k \in \mathbb{N} 				\rbrace $. Potom ak $ Z $ nie je mocninou prvočísla, tak jazyk $ L_Z $ je rozložitelný.
	\end{theorem}

\begin{proof}
\begin{figure}[H]
\centering
\begin{tikzpicture}[->,>=stealth',shorten >=1pt,auto,node distance=2cm,
                    semithick]
   \node[state,initial,accepting] 	(0) 						{$q_0$}; 
   \node[state] 					(1)		[right of=0] 		{$q_1$}; 
   \node							(dots1)	[right of=1]		{\ldots};
   \node[state]	(z-1) 						[right of=dots1] 	{$q_{Z-1}$};
   
   \path[->]
    (0) edge node {a} (1)  
    (1) edge node {a} (dots1)
    (dots1) edge node {a} (z-1)
    (z-1) edge [bend right] node [swap] {a} (0)
    ;
\end{tikzpicture}
\end{figure}

\begin{itemize}
\item $ p_{1}^{m_1}p_{2}^{m_2}...p_{r}^{m_r} $ je prvočíselný rozklad čísla $ Z $
\item automaty $ A_1^Z,A_2^Z $ tvoriace rozklad akceptujú jazyky $ L(A_1^Z) = \lbrace a^{kp_{1}^{m_1}} | k \in \mathbb{N} \rbrace $ a $ L(A_2^Z) = \lbrace a^{kp_{2}^{m_2}...p_{r}^{m_r}} | k \in \mathbb{N} \rbrace $
\end{itemize}

\end{proof}

\end{frame}


\begin{frame}
\frametitle{Jazyky založené na dĺžke slov}

\begin{theorem}
Pre $ n \geq 1 $ a $ p $ je prvočíslo definujeme $ L_{p^n} = \lbrace a^{kp^{n}} | k \in \mathbb{N} \rbrace $. Potom je jazyk $ L_{p^n} $ nerozložitelný.
\end{theorem}

\begin{proof}
\begin{figure}[H]
\centering
\begin{tikzpicture}[->,>=stealth',shorten >=1pt,auto,node distance=2cm,
                    semithick]
   \node[state,initial,accepting] 	(0) 						{$q_0$}; 
   \node[state] 					(1)		[right of=0] 		{$q_1$}; 
   \node							(dots1)	[right of=1]		{\ldots};
   \node[state]	(z-1) 						[right of=dots1] 	{$q_{p^n-1}$};
   
   \path[->]
    (0) edge node {a} (1)  
    (1) edge node {a} (dots1)
    (dots1) edge node {a} (z-1)
    (z-1) edge [bend right] node [swap] {a} (0)
    ;
\end{tikzpicture}
\end{figure}

\begin{itemize}
\item sporom, založený na pumpovaní časti slova $ a^{p^n} $ v automatoch v netriviálnom rozklade a algebraických vlastnostiach následne vyplývajúcich
\end{itemize}

\end{proof}

\end{frame}


\begin{frame}
\frametitle{Uzáverové vlastnosti tried nedeterministicky rozložiteľných a nedeterministicky nerozložteľných jazykov}

\begin{itemize}
\item Nepekné uzáverové vlastnosti
\end{itemize}

\begin{center}
\begin{tabular}{ l ? c | c | c | c | c }
     & $ \cap $ & $ \cup $ & . & $ h $ & $ h^{-1} $ \\
  \specialrule{.2em}{.1em}{.1em}
  R  & \xmark & \xmark & \xmark & \xmark & ? \\
  \hline
  NR & \xmark & \xmark & ? & \xmark & \xmark \\
\end{tabular}
\end{center}

\begin{itemize}
\item zatial sme nepojali ani podozrenie, že by niektorá z tried mohla byť na niečo rozumné uzavretá
\end{itemize}

\end{frame}


\begin{frame}
\frametitle{Príliš malé NKA}

\begin{theorem}
\label{thm:too_small_nsc}
Nech $ L $ je jazyk, pričom $ nsc(L) \leq 2 $. Potom $ L $ je nerozložiteľný.
\end{theorem}

\begin{proof}
\begin{itemize}
\item jednostavové NKA dokážu iba $ \emptyset, \lbrace \varepsilon \rbrace, \Sigma^* $
\item $ \emptyset \subset \lbrace \varepsilon \rbrace \subset \Sigma^* $
\end{itemize}
\end{proof}

\end{frame}


\begin{frame}
\frametitle{Vypchávkové jazyky}

\begin{theorem}
\label{thm:new_symbol_in_language}
Nech $ L \in \mathscr{R} $ a $ b \notin \Sigma_L $. Definujeme homomorfizmus $ h_b: \Sigma_L \cup \lbrace b \rbrace \rightarrow \Sigma_L $ nasledovne - $ h_b(b) = \varepsilon, \; \forall a \in \Sigma_L: h_b(a) = a $. Potom $ L $ je rozložiteľný práve vtedy, keď $ h_{b}^{-1}(L) $ je rozložiteľný
\end{theorem}

\end{frame}


\begin{frame}
\frametitle{Deterministická vs. nedeterministická rozložiteľnosť}

\begin{theorem}
Existuje nedeterministicky nerozložiteľný deterministicky rozložiteľný regulárny jazyk.
\end{theorem}

\begin{proof}
\begin{itemize}
\item rozdielový jazyk je $ (\lbrace a \rbrace \lbrace a,b \rbrace \lbrace a \rbrace \lbrace a,b \rbrace)^* $
\item nedeterministicky nerozložiteľný opäť pomocou pumpovania
\end{itemize}
\end{proof}

\end{frame}


\begin{frame}
\frametitle{Deterministická vs. nedeterministická rozložiteľnosť}

\begin{proof}

\begin{figure}[H]
\centering
\begin{tikzpicture}[scale=0.65,transform shape,->,>=stealth',shorten >=1pt,auto,node distance=2.5cm,
                    semithick]
   \node[state,initial,accepting] 	(0) 					{$q_0$}; 
   \node[state] 					(1)		[right of=0] 	{$q_1$}; 
   \node[state]						(2)		[right of=1]	{$q_2$};
   \node[state]						(3) 	[right of=2] 	{$q_3$};
   \node[state]						(T) 	[below of=1] 	{$q_T$};
   
   \path[->]
    (0) edge [bend right] node {a} 	(1)  
    (1) edge [bend right] node {a,b} (2)
    (2) edge [bend right] node {a} (3)
    (3) edge [bend right] node [swap] {a,b} (0)
    (0) edge node {b} (T)
    (2) edge node {b} (T)
    (T) edge [loop right] node {a,b} ()
    ;
\end{tikzpicture}
\end{figure}

\begin{figure}[H]
\centering
\begin{tikzpicture}[scale=0.65,transform shape,->,>=stealth',shorten >=1pt,auto,node distance=2.5cm,
                    semithick]
   \node[state,initial,accepting] 	(0) 					{$q_0$}; 
   \node[state] 					(1)		[right of=0] 	{$q_1$}; 
   \node[state]						(2)		[right of=1]	{$q_2$};
   \node[state]						(3) 	[right of=2] 	{$q_3$};
   
   \path[->]
    (0) edge [bend right] node {a,b} (1)  
    (1) edge [bend right] node {a,b} (2)
    (2) edge [bend right] node {a,b} (3)
    (3) edge [bend right] node [swap] {a,b} (0)
    ;
\end{tikzpicture}
\begin{tikzpicture}[scale=0.65,transform shape,->,>=stealth',shorten >=1pt,auto,node distance=2.5cm,
                    semithick]
   \node[state,initial,accepting] 	(0) 					{$q_0$}; 
   \node[state] 					(1)		[right of=0] 	{$q_1$}; 
   \node[state]						(T) 	[below right of=0] 	{$q_T$};
   
   \path[->]
    (0) edge [bend right] node {a} 	(1)  
    (1) edge [bend right] node [swap] {a,b} (0)
    (0) edge node [swap] {b} (T)
    (T) edge [loop right] node {a,b} ()
    ;
\end{tikzpicture}

\begin{itemize}
\item Rozložiteľnosť je spôsobená nutnosťou trash-stavu v DKA
\end{itemize}
\end{figure}

\end{proof}

\end{frame}

\begin{frame}
\frametitle{Charakterizácia singleton jazykov}

\begin{theorem}
Nech $ w \in \Sigma^* $ je slovo a $ L = \lbrace w \rbrace $. Potom $ L $ je rozložiteľný práve vtedy, keď $ w = w_1abw_2 $ pre nejaké $ a,b \in \Sigma$ a $ w_1,w_2 \in \Sigma^* $
\end{theorem}

\begin{proof}
\begin{itemize}
\item $ L= \lbrace a^n \rbrace $ je nerozložiteľný
\end{itemize}

\begin{figure}[H]
\centering
\begin{tikzpicture}[scale=0.65,transform shape,->,>=stealth',shorten >=1pt,auto,node distance=2.5cm,
                    semithick]
   \node[state,initial] 			(0) 					{$q_0$}; 
   \node[state] 					(w1)	[right of=0] 	{$q_{w_1}$}; 
   \node[state]						(a)		[right of=w1]	{$q_{a}$};
   \node[state]						(b) 	[right of=a] 	{$q_{b}$};
   \node[state,accepting]			(w2) 	[right of=b] 	{$q_{w_2}$};
   
   \path[->]
    (0) edge [bend left] node {$ w_1 $} (w1)  
    (w1) edge [bend left] node {a} (a)
    (a) edge [bend left] node {b} (b)
    (b) edge [bend left] node {$ w_2 $} (w2)
    ;
\end{tikzpicture}

\begin{tikzpicture}[scale=0.65,transform shape,->,>=stealth',shorten >=1pt,auto,node distance=2.5cm,
                    semithick]
   \node[state,initial] 			(0) 					{$q_0$}; 
   \node[state] 					(w1a)	[right of=0] 	{$q_{w_1a}$}; 
   \node[state]						(b) 	[right of=w1a] 	{$q_{b}$};
   \node[state,accepting]			(w2) 	[right of=b] 	{$q_{w_2}$};
   
   \path[->]
    (0) edge [bend left] node {$ w_1 $} (w1a)  
    (w1a) edge [bend left] node {b} (b)
    (b) edge [bend left] node {$ w_2 $} (w2)
    (w1a) edge [loop above] node {a} ()
    ;
\end{tikzpicture}

\begin{tikzpicture}[scale=0.65,transform shape,->,>=stealth',shorten >=1pt,auto,node distance=2.5cm,
                    semithick]
   \node[state,initial] 			(0) 					{$q_0$}; 
   \node[state] 					(w1)	[right of=0] 	{$q_{w_1}$}; 
   \node[state]						(ab) 	[right of=w1] 	{$q_{ab}$};
   \node[state,accepting]			(w2) 	[right of=ab] 	{$q_{w_2}$};
   
   \path[->]
    (0) edge [bend left] node {$ w_1 $} (w1)  
    (w1) edge [bend left] node {a} (ab)
    (ab) edge [bend left] node {$ w_2 $} (w2)
    (ab) edge [loop above] node {b} ()
    ;
\end{tikzpicture}
\end{figure}

\end{proof}

\end{frame}


\begin{frame}
\frametitle{Predošlý výskum}

\begin{itemize}
\item Idea skúmať užitočnosť informácie vznikla na našej katedre u profesora Rovana
\item Skúmané v súvislosti s deterministickými konečnými automatmi - Gaži (2006)
\item Skúmané v súvislosti s deterministickými zásobníkovými automatmi - Labath (2010)
\item Náš prínos je hlavne v otvorení témy v súvislosti s nedeterminizmom
\end{itemize}
\end{frame}

\begin{frame}
\frametitle{Srdečné poďakovanie}

\begin{center}
\Huge
\textbf{Ďakujem za vašu pozornosť!}
\end{center}

\end{frame}


\end{document}