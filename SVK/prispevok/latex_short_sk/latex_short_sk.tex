% !Mode:: "TeX::UTF-8"

\documentclass{svk_short_sk}
%% ak pisete po anglicky, pouzijete namiesto horneho riadku
%% \documentclass{svk_short_en}

\begin{document}
\title{Ako formátovať ŠVK príspevok}

\author{Jozef Mrkvička\inst{1}
\email{mrkvicka@st.fmph.uniba.sk}
\and 
Tomáš Vinař\inst{1}
\email{vinar@fmph.uniba.sk}}
%% vsimnite si, ze u autorov sa nepisu tituly
%% prikaz \inst sluzi ako odkaz do zoznamu institucii
%% (vid. nizsie)

%% skolitela nepiste medzi autorov, ale v tejto casti
%% ak praca nema skolitela, jednoducho vynechajte
\supervisor{Tom\'a\v{s} Plachetka\inst{2}
\email{plachetk@dcs.fmph.uniba.sk}}

%% nasleduje kratka verzia nazvu clanku a 
%% zoznam autorov (bez krstnych mien)
%% tieto informacie sa zobrazuju v hlavicke
\titlerunning{Ako formátovať ŠVK príspevok}
\authorrunning{Plachetka and Vinař}

\institute{
Katedra aplikovanej informatiky,
FMFI UK,
Mlynská Dolina
842~48~Bratislava
\and
Katedra informatiky,
FMFI UK,
Mlynská Dolina
842~48~Bratislava}

\maketitle

Tento článok je predlohou pre formátovanie príspevku pre ŠVK.
Používa \LaTeX\ štýl {\tt svk\_short\_sk.cls}.
\emph{Nemeňte tento štýl, veci súvisiace s fontami,
veľkosťou stránky, číslovaním strán a podobne.} Krátky príspevok (rozšírený 
abstrakt) obvykle nie je štruktúrovaný do sekcií a 
neobsahuje časť \emph{Abstrakt}.

Nech $S=[s_{ij}]$ ($1\leq i,j\leq n$) je $(0,1,-1)$-matica
veľkosti $n$. Potom $S$ je {\em znamienkovo-nesingulárna matica}
(SNS-matrix), ak každá reálna matica so znamienkovým vzorom
matice $S$ je nesingulárna. V súčasnosti bol silný záujem
o konštrukciu a charakterizáciu
SNS-matíc \cite{bs}, \cite{klm}. Záujem bol tiež o štúdium silných foriem
znamienkovej nesingularity \cite{djd}. V tomto článku ponúkame
nové zovšeobecnenie SNS-matíc a skúmame niektoré ich základné vlastnosti.

V tomto článku sa zaoberáme výpočtom integrálov nasledujúcich druhov:
\begin{equation}
\int_a^b \left( \sum_i E_i B_{i,k,x}(t) \right)
         \left( \sum_j F_j B_{j,l,y}(t) \right) dt,\label{problem}
\end{equation}
\begin{equation}
\int_a^b f(t) \left( \sum_i E_i B_{i,k,x}(t) \right) dt,\label{problem2}
\end{equation}
kde $B_{i,k,x}$ je $i$-ty B-splajn stupňa $k$ definovaný v uzloch
$x_i, x_{i+1}, \ldots, x_{i+k}$.
Budeme predpokladať, že B-splajny sú normalizované tak, že ich integrál je 
jednotkový.
Splajny môžu byť rôznych stupňov a môžu byť definované v navzájom rôznych 
postupnostiach uzlov
$x$ and $y$.
Limity integrácie budú často
od $-\infty$ po
$+\infty$. Všimnite si, že (\ref{problem}) je špeciálnym prípadom
(\ref{problem2}),
kde $f(t)$ je splajn.

S použitím súčinovej topológie na  $R^{m \times m} \times
R^{n \times n}$ s indukovaným súčinom 
\begin{equation}
\langle (A_{1},B_{1}), (A_{2},B_{2})\rangle := \langle A_{1},A_{2}\rangle
+ \langle B_{1},B_{2}\rangle,\label{eq2.10}
\end{equation}
vypočítame Fr\'{e}chetovu deriváciu $F$ nasledujúcim spôsobom:
\begin{eqnarray}
 F'(U,V)(H,K) &=& \langle R(U,V),H\Sigma V^{T} + U\Sigma K^{T}\nonumber\\
             && - P(H\Sigma V^{T} + U\Sigma K^{T})\rangle \nonumber \\
         &=& \langle R(U,V),H\Sigma V^{T} + U\Sigma K^{T}\rangle\nonumber \\
&=& \langle R(U,V)V\Sigma^{T},H\rangle + \nonumber\\
  &&    \langle \Sigma^{T}U^{T}R(U,V),K^{T}\rangle.    \label{eq2.11}
\end{eqnarray}


\begin{theorem} 
\label{th:prop} 
Dvojica matíc $(S,C)$ je {\rm SNS}-maticový pár, ak všetky nenulové
koeficienty jeho charakteristického polynómu majú rovnaké znamienko a
ak aspoň jeden z koeficientov je nenulový.
\end{theorem} 

Pre SNS-maticové páry $(S,C)$ s $C=O$ je charakteristický polynóm homogénny,
stupňa $n$. V tom prípade je Veta~\ref{th:prop} triviálnym dôsledkom 
vlastností SNS-matíc.
 
\bibliographystyle{apalike}
\bibliography{references}

%% citacie ulozte do suboru references.bib
%% na populaciu zoznamu literatury pouzite program
%%
%% bibtex references
%%
%% po ktorom je potrebne dokument znova zlatexovat

\end{document}
