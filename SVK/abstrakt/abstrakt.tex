%sample file for Modelica 2011 Abstract page

\documentclass[11pt,a4paper]{article}
\usepackage{graphicx}
% uncomment according to your operating system:
% ------------------------------------------------
%\usepackage[latin1]{inputenc}    %% european characters can be used (Windows, old Linux)
\usepackage[utf8]{inputenc}     %% european characters can be used (Linux)
%\usepackage[applemac]{inputenc} %% european characters can be used (Mac OS)
% ------------------------------------------------
\usepackage[T1]{fontenc}   %% get hyphenation and accented letters right
\usepackage{mathptmx}      %% use fitting times fonts also in formulas
% do not change these lines:
\pagestyle{empty}                %% no page numbers!
\usepackage[left=35mm, right=35mm, top=15mm, bottom=20mm, noheadfoot]{geometry}
%% please don't change geometry settings!
\usepackage[slovak]{babel} 

% begin the document
\begin{document}
\thispagestyle{empty}

\title{\textbf{Prídavná informácia a zložitosť nedeterministických konečných automatov} \\ 
(predbežný abstrakt)}
\author{Šimon Sádovský\quad Branislav Rovan\\
Fakulta matematiky, fyziky a informatiky, Univerzita Komenského, Bratislava}
\date{} % <--- leave date empty
\maketitle\thispagestyle{empty} %% <-- you need this for the first page

V práci skúmame vplyv prídavnej informácie na zložitosť riešenia problému. Ako výpočtový model sme zvolili nedeterministické konečné automaty a mierou zložitosti je počet stavov. Formalizáciou nášho problému je rozklad nedeterministického konečného automatu na dvojicu nedeterministických konečných automatov takých, že jazyk pôvodného automatu je prienikom jazykov týchto dvoch automatov. Navyše očakávame, že oba tieto automaty budú jednoduchšie ako pôvodný automat. V práci dokazujeme rozložiteľnosť respektíve nerozložiteľnosť konkrétnych regulárnych jazykov. Dokazujeme uzáverové a iné vlastnosti tried nedeterministicky rozložiteľných a nedeterministicky nerozložiteľných regulárnych jazykov. Charakterizujeme vzhľadom na rozložiteľnosť triedu jazykov, ktoré sú tvorené práve jedným slovom. Skúmame jazyky, ktorých minimálny nedeterministický automat je tvorený práve jedným cyklom. Ukazujeme rozdiel medzi nedeterministickou a deterministickou rozložiteľnosťou regulárnych jazykov.

\paragraph*{Kľúčové slová:} nedeterministický konečný automat, rozklad nedeterministického konečného automatu, nedeterministická rozložiteľnosť, prídavná informácia, popisná zložitosť

\end{document}
