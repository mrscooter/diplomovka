% !Mode:: "TeX::UTF-8"
\documentclass[myposter,portrait]{sciposter}

%% uzitocne package
\usepackage{multicol}
\usepackage{color}
\usepackage{graphicx}
\usepackage{tikz}
\usepackage{float}
\usepackage{verbatim}
\usetikzlibrary{automata,arrows,topaths}
\usepackage{cite}

%% znaky s diakritikou
\usepackage[utf8]{inputenc}
\usepackage[T1]{fontenc}
\usepackage[slovak]{babel} % slovenske delenie slov
\usepackage{mathrsfs} 
\usepackage{amsfonts}
\usepackage{amsthm}
\usepackage{enumitem}

%% definicia farieb
\definecolor{mainCol}{rgb}{0.91,0.82,0.74} % farba pozadia posteru
%\definecolor{mainCol}{rgb}{0.73,0.33,0.83} % liafova
\definecolor{sectionCol}{rgb}{0,0,0} % farba nadpisu
\definecolor{textCol}{rgb}{0.2,0,0} % farba hlavneho textu
\definecolor{BoxCol}{rgb}{1,1,0.8} % farba boxu okolo nadpisov
%\definecolor{BoxCol}{rgb}{1,1,1} % farba boxu okolo nadpisov

\def\mysection#1{
{\color{sectionCol}\section*{\sc\bfseries #1}}}

\newtheorem*{notation}{Označenie}
\newtheorem*{definition}{Definícia}
\newtheorem*{theorem}{Veta}

\begin{document}
\setlength{\logowidth}{20cm}
\setlength{\titlewidth}{\textwidth}
\addtolength{\titlewidth}{-\logowidth}
\rightlogo[0.9]{fmfilogo-farebne}
\useleftlogofalse

\color{textCol}

\title{Prídavná informácia a zložitosť nedeterministických konečných automatov}
\author{Šimon Sádovský$^1$\\
        Školiteľ: Branislav Rovan$^1$}
\institute{%
$^1$ Katedra informatiky, 
FMFI UK, Mlynská Dolina, 842~48~Bratislava
}
\maketitle

\begin{multicols*}{3}

\mysection{Motivácia problému}
V práci skúmame vplyv prídavnej informácie na zložitosť riešenia problému. Ako
výpočtový model sme zvolili nedeterministické konečné automaty a mierou zložitosti je
počet stavov. Voľne povedané, ak automatu garantujeme, že vstup, ktorý ide rozpoznávať patrí do nejakého poradného jazyka, vieme tým dosiahnuť, že na rozpoznávanie pôvodného jazyka stačí automat menšej zložitosti? Uveďme jeden príklad. Uvažujme, že chceme rozpoznávať jazyk $ \lbrace w \in \lbrace a \rbrace^* \; | \; |w| \equiv 0 \; (mod \; 6) \rbrace $ a chceme ho rozpoznávať nedeterministickým konečným automatom. Ľahko vidno, že minimálny NKA pre tento jazyk má 6 stavov. Čo ak vopred vieme, že dĺžka vstupu je deliteľná tromi? Vtedy nám stačí vziať NKA s dvomi stavmi. Analogický problém bol skúmaný pre deterministické konečné automaty v \cite{Gazi} a pre deterministické zásobníkové automaty \cite{Labath}.

\mysection{Formalizácia problému}

Formalizáciou tohto problému je hľadanie rozkladov nedeterministických automatov.
\\

\begin{notation}
Počet stavov ľubovolného konečného automatu $ A $ označujeme $ \#_S(A) $.\\
\end{notation}

\begin{definition}
Nech $ A $ je nedeterministický konečný automat. Potom dva nedeterministické konečné automaty $ A_1, A_2 $ také, že $ L(A)=L(A_1) \cap L(A_2) $ nazveme \textbf{rozklad automatu} $ A $. Ak navyše platí $ \#_S(A_1) < \#_S(A)$ a $ \#_S(A_2) < \#_S(A) $, nazývame tento rozklad \textbf{netriviálny}. Ak existuje netriviálny rozklad automatu $ A $, tak automat $ A $ nazývame \textbf{rozložiteľný}.\\
\end{definition}

Vlastnosť rozložiteľnosti sa dá prirodzene rozšíriť aj na vlastnosť regulárnych jazykov.\\

\begin{definition}
\label{def:nedeterministic_decomposability_of_language}
Nech $ L \in \mathscr{R} $ a $ A $ je nejaký minimálny NKA pre jazyk L. \textbf{Jazyk} $ L $ nazývame \textbf{nedeterministicky rozložiteľný} práve vtedy, keď je automat $ A $ rozložiteľný.
\end{definition}

%\columnbreak 

\mysection{Jazyky založené na module dĺžky slova}

Prirodzenou schopnosťou konečných automatov je v cykle rátať zvyšok po delení dĺžky slova. Pokiaľ je toto jediná vec, ktorú NKA robí, jedná sa o jazyky typu $ \lbrace a^{kn} | k \in \mathbb{N} \rbrace $. Minimálne NKA akceptujúce takéto jazyky vyzerajú nasledovne.

\begin{figure}[H]
\centering
\begin{tikzpicture}[->,>=stealth',shorten >=1pt,auto,node distance=5cm,semithick,accepting/.style={double distance=2.5pt}]
   \node[state,initial,accepting] 	(0) 						{$q_0$}; 
   \node[state] 					(1)		[right of=0] 		{$q_1$}; 
   \node							(dots1)	[right of=1]		{\ldots};
   \node[state]	(n-1) 						[right of=dots1] 	{$q_{n-1}$};
   
   \path[->]
    (0) edge node {a} (1)  
    (1) edge node {a} (dots1)
    (dots1) edge node {a} (n-1)
    (n-1) edge [bend right] node [swap] {a} (0)
    ;
\end{tikzpicture}
\end{figure}

Charakterizujeme tieto jazyky vzhľadom na nedeterministickú rozložiteľnosť.

\begin{theorem}
\label{thm:primes_compounds}
Nech pre $ n \in \mathbb{N}, n > 0 $ je $ L_n = \lbrace a^{kn} \; | \; k \in \mathbb{N} \rbrace $. Potom $ L_n $ je nedeterministicky rozložiteľný práve vtedy, keď $ n $ nie je mocninou prvočísla.
\end{theorem}

\mysection{Jednoslovné jazyky} 
Charakterizujeme vzhľadom na nedeterministickú rozložiteľnosť jazyky, ktoré sú tvorené práve jedným slovom.

\bigskip

\begin{theorem}
\label{thm:singleton_characterization}
Nech $ L = \lbrace w \rbrace $. Potom je $ L $ nedeterministicky rozložiteľný práve vtedy, keď $ w $ obsahuje aspoň dva rôzne symboly.
\end{theorem}

\mysection{Príliš malé automaty}

Ukazujeme, že príliš malé NKA sú nerozložiteľné.
\bigskip

\begin{notation}
Počet stavov minimálneho NKA akceptujúceho jazyk $ L $ označujeme \em{}$ nsc(L) $\em{}.
\end{notation}

\bigskip

\begin{theorem}
\label{thm:too_small_nsc}
Nech $ L \in \mathscr{R} $, pričom $ nsc(L) \leq 2 $. Potom $ L $ je nedeterministicky nerozložiteľný.
\end{theorem}

\mysection{Rozdiel medzi deterministickou a nedeterministickou rozložiteľnosťou}

Podstatnú časť našej práce tvorí hľadanie takzvaných rozdielových jazykov, ktoré by dokazovali, že pojem rozložiteľnosti regulárneho jazyka je rôzny ak uvažujeme deterministické resp. nedeterministické automaty. Intuícia našepkáva, že ak to pôjde, tak by to mohlo ísť skôr tak, že nájdeme jazyk ktorý je deterministicky nerozložiteľný a zároveň nedeterministicky rozložiteľný (očakávali sme, že v rozklade ušetrí nedeterminizmus stavy). Avšak podarilo sa nám nájsť rozdielové jazyky, pri ktorých to je opačne. Prvým našim výsledkom v tomto smere je nasledujúce tvrdenie.

\bigskip

\begin{theorem}
Jazyk $ (\lbrace a \rbrace \lbrace a,b \rbrace \lbrace a \rbrace \lbrace a,b \rbrace)^* $ je deterministicky rozložiteľný a súčasne nedeterministicky nerozložiteľný.
\end{theorem}

\bigskip

Chybou krásy tohoto výsledku bolo, že rozklad funguje iba vďaka požiadavke na úplnosť prechodovej funkcie v definícii DKA, vďaka čomu musí mať DKA pre rozdielový jazyk odpadový stav. Podarilo sa nám však nájsť nekonečnú postupnosť jazykov takú, ktorá tento problém už nemá.

\bigskip

\begin{theorem}
\label{thm:ndet_vs_det_diff_big}
Existuje postupnosť jazykov $  (L_i)_{i=2}^{\infty} $, taká, že platí:
\begin{enumerate}[label=(\alph*)]
\item Jazyk $ L_i $ je nedeterministicky nerozložiteľný a súčasne deterministicky rozložiteľný pre ľubovolné $ i \in \mathbb{N}, i \geq 2 $.
\item Nech pre ľubovolné $ i \in \mathbb{N}, i \geq 3 $ je $ A_i $ minimálny DKA akceptujúci $ L_i $. Potom existuje taký rozklad $ A_i $ na $ A_1^i $ a $ A_2^i $, že platí $ \#_S(A_1^i)=\#_S(A_2^i)=\frac{\#_S(A_i)+3}{2} $.
\end{enumerate}
\end{theorem}

\bigskip

Definujme postupnosť jazykov $ (L_i)_{i=2}^{\infty} $ nasledovne: $ L_i = (\lbrace a^{i-1} \rbrace \lbrace b \rbrace^* \lbrace a,b \rbrace)^* $ pre ľubovolné $ i \geq 2 $. Táto postupnosť ešte nie je tá, ktorú hľadáme. Tú, ktorú hľadáme, však dostaneme z $ P $ vybratím niektorých (nekonečne veľa) jej členov. Na príklade jazyka $ L_4 $ možno vidieť ako vyzerá minimálny NKA pre jazyk $ L_i $.

\begin{figure}[H]
\centering
\begin{tikzpicture}[->,>=stealth',shorten >=1pt,auto,node distance=4cm,
                    semithick,accepting/.style={double distance=2.5pt}]
	\node[state,initial,accepting] (0)					{$q_0$};
   	\node[state] 					(1)	[right of=0]	{$q_1$};
   	\node[state] 					(2)	[below of=1]	{$q_2$};
   	\node[state] 					(3)	[left of=2]		{$q_3$};
   	
   	\path[->]
     (0) edge [bend left] node {$ a $} (1)
     (1) edge [bend left] node {$ a $} (2)
     (2) edge [bend left] node {$ a $} (3)
     (3) edge [bend left] node {$ a,b $} (0)
     (3) edge [loop below] node {$ b $} ()
     ;
\end{tikzpicture}
\end{figure}

Tento NKA je pre nekonečne veľa $ i \geq 2 $ nerozložiteľný, t.j. $ L_i $ je nedeterministicky nerozložiteľný pre nekonečne veľa $ i \geq 2 $. Minimálny DKA pre $ L_i $ dostaneme štandardným prevodom z NKA. Ilustujeme na príklade pre $ L_4 $.

\begin{figure}[H]
\centering
\begin{tikzpicture}[->,>=stealth',shorten >=1pt,auto,node distance=4cm,
                    semithick,accepting/.style={double distance=2.5pt}]
	\node[state,initial,accepting] (0)						{$0$};
   	\node[state] 					(1)		[below of=0]	{$1$};
   	\node[state] 					(2)		[right of=1]	{$2$};
   	\node[state] 					(3)		[above of=2]	{$3$};
	\node[state,accepting] 			(30)	[right of=3]	{$3,0$};
   	\node[state,accepting]			(01)	[right of=30]	{$0,1$};
   	\node[state] 					(12)	[below of=01]	{$1,2$};
   	\node[state] 					(23)	[left of=12]	{$2,3$};
   	
   	\path[->]
     (0) edge [bend right] node [swap] {$ a $} (1)
     (1) edge [bend right] node [swap] {$ a $} (2)
     (2) edge [bend right] node [swap] {$ a $} (3)
     (3) edge [bend right] node [swap] {$ a $} (0)
     (3) edge [bend left] node {$ b $} (30)
     (30) edge [bend left] node {$ a $} (01)
     (01) edge [bend left] node {$ a $} (12)
     (12) edge [bend left] node {$ a $} (23)
     (23) edge [bend left] node [swap] {$ a,b $} (30)
     (30) edge [loop above] node {$ b $} ()
     ;
\end{tikzpicture}
\end{figure}

Myšlienkou rozkladu je, že každý z automatov v rozklade \glqq{}vyrieši\grqq{} len jeden z cyklov pôvodného DKA.

\begin{figure}[H]
\centering
\begin{tikzpicture}[->,>=stealth',shorten >=1pt,auto,node distance=3.5cm,
                    semithick,accepting/.style={double distance=2.5pt}]
	\node[state,initial,accepting] (C1)					{$C1$};
	\node[state,accepting] 			(30)	[right of=C1]	{$3,0$};
   	\node[state,accepting]			(01)	[above right of=30]	{$0,1$};
   	\node[state] 					(12)	[below right of=01]	{$1,2$};
   	\node[state] 					(23)	[below left of=12]	{$2,3$};
   	
   	\path[->]
     (C1) edge [loop above] node {$ a $} ()
     (C1) edge [bend left] node {$ b $} (30)
     (30) edge [bend left] node {$ a $} (01)
     (01) edge [bend left] node {$ a $} (12)
     (12) edge [bend left] node {$ a $} (23)
     (23) edge [bend left] node {$ a,b $} (30)
     (30) edge [loop right] node {$ b $} ()
     ;
\end{tikzpicture}

\begin{tikzpicture}[->,>=stealth',shorten >=1pt,auto,node distance=3.5cm,
                    semithick,accepting/.style={double distance=2.5pt}]
	\node[state,initial above ,accepting] (0)				{$[0]$};
   	\node[state] 					(1)		[below left of=0]	{$[1]$};
   	\node[state] 					(2)		[below right of=1]	{$[2]$};
   	\node[state] 					(3)		[above right of=2]	{$[3]$};
	\node[state,accepting] 			(C2)	[right of=3]		{$C2$};
   	
   	\path[->]
     (0) edge [bend right] node [swap] {$ a $} (1)
     (1) edge [bend right] node [swap] {$ a $} (2)
     (2) edge [bend right] node [swap] {$ a $} (3)
     (3) edge [bend right] node [swap] {$ a $} (0)
     (3) edge [bend left] node {$ b $} (C2)
     (C2) edge [loop above] node {$ a,b $} ()
     ;
\end{tikzpicture}
\end{figure}

Jazyk $ L_i $ je deterministicky rozložiteľný pre ľubovolné $ i \geq 2 $. Teda sme našli nekonečnú postupnosť s hľadanými vlastnosťami.

%% zoznam literatury
\bibliographystyle{apalike}
\bibliography{literatura}

\end{multicols*}
\end{document}

