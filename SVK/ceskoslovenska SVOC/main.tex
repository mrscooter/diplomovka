%diplomovka
\documentclass[12pt, oneside]{book}
\usepackage[a4paper,top=2.5cm,bottom=2.5cm,left=3.5cm,right=2cm]{geometry}
\usepackage[utf8]{inputenc}
\usepackage[T1]{fontenc}
\usepackage{graphicx}
\usepackage{cite}
\usepackage{url}
\usepackage{amsthm}
\usepackage{amsmath}
\usepackage{mathrsfs}
\usepackage{amssymb}
\usepackage{enumitem}
\usepackage{tikz}
\usepackage{float}
\usepackage{verbatim}
\usetikzlibrary{automata,arrows,topaths,positioning}
\usepackage[slovak]{babel} % vypnite pre prace v anglictine
\linespread{1.25} % hodnota 1.25 by mala zodpovedat 1.5 riadkovaniu

\newtheorem{theorem}{Veta}[section]
\newtheorem{lemma}{Lema}[section]
\newtheorem{example}{Príklad}[section]
\newtheorem{definition}{Definícia}[section]
\newtheorem{notation}{Označenie}[section]
\newtheorem{note}{Poznámka}[section]
\newtheorem{corollary}{Dôsledok}[section]
\newtheorem{proposition}{Tvrdenie}[section]

% -------------------
% --- Definicia zakladnych pojmov
% --- Vyplnte podla vasho zadania
% -------------------
\def\mfrok{2017}
\def\mfnazov{Prídavná informácia a zložitosť nedeterministických konečných automatov}
\def\mftyp{práca pre súťaž SVOČ}
\def\mfautor{Bc. Šimon Sádovský}
\def\mfskolitel{prof. RNDr. Branislav Rovan, PhD.}

%ak mate konzultanta, odkomentujte aj jeho meno na titulnom liste
\def\mfkonzultant{tit. Meno Priezvisko, tit. }  

\def\mfmiesto{Bratislava, \mfrok}

%aj cislo odboru je povinne a je podla studijneho odboru autora prace
\def\mfodbor{2508 Informatika} 
\def\program{ Informatika }
\def\mfpracovisko{ Katedra informatiky }

\begin{document}     

% -------------------
% --- Obalka ------
% -------------------
\thispagestyle{empty}

\begin{center}
\sc\large
Univerzita Komenského v Bratislave\\
Fakulta matematiky, fyziky a informatiky

\vfill

{\LARGE\mfnazov}\\
\mftyp
\end{center}

\vfill

{\sc\large 
\noindent \mfrok \\ \mfautor \\ Školiteľ: \mfskolitel
}

\eject % EOP i
% --- koniec obalky ----

\frontmatter

% -------------------
%   Poďakovanie - nepovinné
% -------------------
\setcounter{page}{3}
\newpage 
~

\vfill
{\bf Poďakovanie:}
Sú momenty v živote, keď sa človeku podarí niečo, čo vníma, že nie je celkom obyčajné, každodenné. Niečo, čo vyžadovalo vynaložiť nemalé úsilie a nútilo ho objaviť v sebe veci, o ktorých by nepovedal, že v ňom sú. Pre mňa je niečim takýmto práve táto práca. O to viac prežívam konkrétnu vďačnosť voči konkrétnym osobám, bez ktorých by som nemal najmenšiu šancu niečo takéto vytvoriť. Uvedomujem si, že som mohol nadviazať na prácu velikánov našej civilizácie, počínajúc Aristotelom, pokračujúc Eulerom, Newtonom, Leibnitzom, Gödelom, Turingom a mnohými ďalšími, ktorých práca je pre mňa niečím naozaj hlboko krásnym a inšpirujúcim. Bez mojich rodičov by som to tiež ďaleko nedotiahol a patrí im vďaka za to, že ma vytrvalo podporovali v mojich štúdiách a nie iba v nich. Takisto som veľmi vďačný všetkým mojím priateľom. Najviac Janke, Makiovi, Pallimu a Daliborovi, ktorí ma neraz naozaj veľmi podržali. Priatelia, ste pre mňa vzácni. Ďakujem tiež všetkým mojím spoluhráčom z eRka, UPeCe a Katarínky, s ktorými som strávil a stále trávim krásny čas v dobrovoľníckej službe. Veľkú vďaku chcem vyjadriť Bohu, ktorému som dal v mojom živote šancu a od vtedy mení môj život na niečo nádherné. Vďaka patrí takisto všetkým učiteľom, ktorí ma kedy učili. Chcel by som konkrétne poďakovať mojej prvej pani učiteľke Evke, ktorá mi ako prvá otvorila cestu ku vzdelaniu. Takisto ďakujem výbornej matematikárke Darinke Madleňákovej, ktorá mi na strednej dala kvalitné základy do matematiky. Ďakujem najlepšej triednej Marike Khürovej, ktorá dokázala trpezlivo zvládať všetky moje pubertálne nápady, ktoré občas stáli za to. Tiež ďakujem každému pracovníkovi katedry informatiky, ktorí okrem svojej odbornosti vytvárajú naozaj príjemne kolegiálnu a ľudskú atmosféru na pracovisku. Ako poslednému, ale o to viac, chcem poďakovať vedúcemu tejto práce, pánovi profesorovi Branislavovi Rovanovi. Okrem jeho veľmi cenných odborných rád a usmernení ďakujem za každú jednu kávu, ktorou ma na našich konzultáciách ponúkol, a za to, že je vždy tak normálne prirodzene ľudský. Takže ešte raz, vďaka!

% --- Koniec poďakovania

% -------------------
%   Abstrakt - Slovensky
% -------------------
\newpage 
\section*{Abstrakt}

V práci skúmame vplyv prídavnej informácie na zložitosť riešenia problému. Ako výpočtový model sme zvolili nedeterministické konečné automaty a mierou zložitosti je počet stavov. Formalizáciou nášho problému je rozklad nedeterministického konečného automatu na dvojicu nedeterministických konečných automatov takých, že jazyk pôvodného automatu je prienikom jazykov týchto dvoch automatov. Navyše očakávame, že oba tieto automaty budú jednoduchšie ako pôvodný automat. V práci dokazujeme rozložiteľnosť, respektíve nerozložiteľnosť konkrétnych regulárnych jazykov. Dokazujeme uzáverové a iné vlastnosti tried nedeterministicky rozložiteľných a nedeterministicky nerozložiteľných regulárnych jazykov. Charakterizujeme triedu jazykov tvorených jedným slovom vzhľadom na rozložiteľnosť. Skúmame jazyky, ktorých minimálny nedeterministický automat je tvorený práve jedným cyklom. Ukazujeme rozdiel medzi nedeterministickou a deterministickou rozložiteľnosťou regulárnych jazykov.

\paragraph*{Kľúčové slová:} nedeterministický konečný automat, rozklad nedeterministického konečného automatu, nedeterministická rozložiteľnosť, prídavná informácia, popisná zložitosť
% --- Koniec Abstrakt - Slovensky


% -------------------
% --- Abstrakt - Anglicky 
% -------------------
\newpage 
\section*{Abstract}

We study the effect of supplementary information on the complexity of problem solution. We have chosen nondeterministic finite automaton as the computational model and we measure the complexity by the number of states. We formalize our problem via decomposition of nondeterministic finite automaton into two nondeterministic finite automata, such that the language accepted by the original automaton is the intersection of languages accepted by this two automata. Moreover, we require both automata in the decomposition to be simpler than the original automaton. We show decomposability and nondecomposability of particular regular languages. We show closure and other properties of classes of nondeterministically decomposable and nondecomposable regular languages. We characterize the class of languages consisting of exactly one word with respect to decomposability. We examine languages accepted by nondeterministic automata consisting of exactly one cycle. We show the difference between nondeterministic and deterministic decomposability of regular languages.


\paragraph*{Keywords:} nondeterministic finite automaton, decomposition of nondeterministic finite automaton, nondeterministic decomposability, supplementary information, descriptional complexity

% --- Koniec Abstrakt - Anglicky

% -------------------
% --- Predhovor - v informatike sa zvacsa nepouziva
% -------------------
%\newpage 
%\thispagestyle{empty}
%
%\huge{Predhovor}
%\normalsize
%\newline
%Predhovor je všeobecná informácia o práci, obsahuje hlavnú charakteristiku práce 
%a okolnosti jej vzniku. Autor zdôvodní výber témy, stručne informuje o cieľoch 
%a význame práce, spomenie domáci a zahraničný kontext, komu je práca určená, 
%použité metódy, stav poznania; autor stručne charakterizuje svoj prístup a svoje 
%hľadisko. 
%
% --- Koniec Predhovor


% -------------------
% --- Obsah
% -------------------

\newpage 

\tableofcontents

% ---  Koniec Obsahu

% -------------------
% --- Zoznamy tabuliek, obrázkov - nepovinne
% -------------------

\newpage 

\listoffigures

% ---  Koniec Zoznamov

\mainmatter


\input uvod.tex

\input definicie.tex

\input rozlozitelneNerozlozitelne.tex

\input detVsNdet.tex

\input vlastnostiRNR.tex

\input zaver.tex

% -------------------
% --- Bibliografia
% -------------------


\newpage	

\backmatter

\thispagestyle{empty}
\nocite{*}
\clearpage

\bibliographystyle{apalike}
\bibliography{literatura}

%Prípadne môžete napísať literatúru priamo tu
%\begin{thebibliography}{5}
 
%\bibitem{br1} MOLINA H. G. - ULLMAN J. D. - WIDOM J., 2002, Database Systems, Upper Saddle River : Prentice-Hall, 2002, 1119 s., Pearson International edition, 0-13-098043-9

%\bibitem{br2} MOLINA H. G. - ULLMAN J. D. - WIDOM J., 2000 , Databasse System implementation, New Jersey : Prentice-Hall, 2000, 653s., ???

%\bibitem{br3} ULLMAN J. D. - WIDOM J., 1997, A First Course in Database Systems, New Jersey : Prentice-Hall, 1997, 470s., 

%\bibitem{br4} PREFUSE, 2007, The Prefuse visualization toolkit,  [online] Dostupné na internete: <http://prefuse.org/>

%\bibitem{br5} PREFUSE Forum, Sourceforge - Prefuse Forum,  [online] Dostupné na internete: <http://sourceforge.net/projects/prefuse/>

%\end{thebibliography}

%---koniec Referencii

% -------------------
%--- Prilohy---
% -------------------

%Nepovinná časť prílohy obsahuje materiály, ktoré neboli zaradené priamo  do textu. Každá príloha sa začína na novej strane.
%Zoznam príloh je súčasťou obsahu.
%
%\addcontentsline{toc}{chapter}{Appendix A}
%\input AppendixA.tex
%
%\addcontentsline{toc}{chapter}{Appendix B}
%\input AppendixB.tex

\end{document}






