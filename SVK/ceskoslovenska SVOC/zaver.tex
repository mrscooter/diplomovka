\chapter*{Záver}
\addcontentsline{toc}{chapter}{Záver}

V práci sme nadviazali na výskum v oblasti skúmania rôznych aspektov informácie. Skúmali sme pojem užitočnosti informácie. Otvorili sme oblasť skúmania užitočnosti prídavnej informácie v kontexte nedeterminizmu. Ako výpočtový model sme zvolili nedeterministické konečné automaty s mierou zložitosti počet stavov. Formalizáciou nášho problému je rozklad nedeterministického konečného automatu. Pojem rozložiteľnosti sme prirodzene rozšírili na regulárne jazyky.
\par
V práci sme dokázali rozložiteľnosť, respektíve nerozložiteľnosť, niekoľkých konkrétnych regulárnych jazykov. Tieto výsledky pomáhajú uchopiť problém rozložiteľnosti a nerozložiteľnosti a pomáhajú vybudovať dôkazové techniky. Tieto výsledky sme následne použili pri skúmaní uzáverových vlastností rozložiteľných, respektíve nerozložiteľných regulárnych jazykov. Dokázali sme, že tieto triedy nie sú uzavreté na žiadnu z bežných operácií. Charakterizovali sme dve podtriedy regulárnych jazykov vzhľadom na rozložiteľnosť. Sú to jazyky pozostávajúce z jedného slova a jazyky tvaru $ \lbrace a^{kn} \; | \; k \in \mathbb{N} \rbrace $. Ukázali sme rozdiel medzi deterministickou a nedeterministickou rozložiteľnosťou. Našli sme nekonečnú postupnosť regulárnych jazykov, ktoré sú nedeterministicky nerozložiteľné a súčasne deterministicky rozložiteľné. Navyše rozklad minimálneho deterministického konečného automatu pre tieto jazyky je taký, že oba deterministické automaty v rozklade majú asi polovicu stavov vzhľadom k pôvodnému automatu.
\par
Možným pokračovaním tejto práce je hľadanie charakterizácií ďalších netriviálnych podtried regulárnych jazykov vzhľadom na nedeterministickú rozložiteľnosť. Veľmi dobrým výsledkom by bolo nájsť charakterizáciu regulárnych jazykov vzhľadom na rozložiteľnosť. Z uzáverových vlastností by sa dala skúmať ešte uzavretosť na reverz a komplement. Zmysluplným pokračovaním je tiež skúmanie rozložiteľnosti nedeterministického konečného automatu ako takého (neuvažovať v kontexte rozložiteľnosti jazyka) tak, že sa pozrieme na jeho definíciu a z nej skúsime usúdiť, či je daný automat rozložiteľný (dalo by sa pozrieť napr. na grafové vlastnosti diagramu daného automatu). Ďalšou možnosťou je skúmanie pojmu rozložiteľnosti pre iné, silnejšie výpočtové modely.