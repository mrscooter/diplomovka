\chapter[Iné výsledky]{Iné výsledky}
\label{kap:properties_of_classes}

Uvidíme čo kde ešte dať tak zatial sem

\section{Porovnanie deterministickej a nedeterministickej rozložiteľnosti regulárnych jazykov}

Zaujímavou otázkou je, či existuje regulárny jazyk taký, že je deterministicky nerozložiteľný a súčasne nedeterminiticky rozložiteľný respektíve deterministicky rozložiteľný a súčasne nedeterminiticky nerozložiteľný. Pred tým, ako uvedieme dosiahnuté výsledky zavedieme definícu deterministického konečného automatu, ktorú budeme používať, nakoľko existuje viacero prístupov k definovaniu deterministických konečných automatov.

\begin{definition}
\textbf{Deterministický konečný automat} je pätica $ (K, \Sigma, \delta, q_0, F) $, kde:
\begin{enumerate}  
\item $ K $ je konečná množina stavov
\item $ \Sigma $ je konečná vstupná abeceda
\item $ q_0 \in K $ je počiatočný stav
\item $ F \subseteq K $ je množina akceptačných stavov
\item $ \delta : K \times \Sigma \rightarrow K $ je prechodová funkcia
\end{enumerate}
\end{definition}

\begin{note}
Deterministický konečný automat sa skrátene označuje DKA.
\end{note}

Poznajúc ako v našom texte definujeme deterministický konečný automat je pre čitateľa so základnými znalosťami v oblasti jasné, ako by boli definované ostatné potrebné pojmy, preto ich definície neuvádzame.

\begin{theorem}
Existuje nedeterministicky nerozložiteľný deterministicky rozložiteľný regulárny jazyk.
\end{theorem}

\begin{proof}
Hľadaným jazykom je jazyk $ L = (\lbrace a \rbrace \lbrace a,b \rbrace \lbrace a \rbrace \lbrace a,b \rbrace)^* $. Ukážeme, že jazyk $ L $ je deterministicky rozložiteľný. Najprv zostrojíme minimálny DKA $ A_{L} $ akceptujúci $ L $. Automat uvádzame pomocou diagramu.

\begin{figure}[H]
\centering
\begin{tikzpicture}[->,>=stealth',shorten >=1pt,auto,node distance=2.5cm,
                    semithick]
   \node[state,initial,accepting] 	(0) 					{$q_0$}; 
   \node[state] 					(1)		[right of=0] 	{$q_1$}; 
   \node[state]						(2)		[right of=1]	{$q_2$};
   \node[state]						(3) 	[right of=2] 	{$q_3$};
   \node[state]						(T) 	[below of=1] 	{$q_T$};
   
   \path[->]
    (0) edge [bend right] node {a} 	(1)  
    (1) edge [bend right] node {a,b} (2)
    (2) edge [bend right] node {a} (3)
    (3) edge [bend right] node [swap] {a,b} (0)
    (0) edge node {b} (T)
    (2) edge node {b} (T)
    (T) edge [loop right] node {a,b} ()
    ;
\end{tikzpicture}
\caption{deterministický konečný automat $ A_{L} $ pre jazyk $ L = (\lbrace a \rbrace \lbrace a,b \rbrace \lbrace a \rbrace \lbrace a,b \rbrace)^* $} \label{fig:DFAa(a|b)a(a|b)^4}
\end{figure}

Ľahko vidno, že $ A_L $ akceptuje práve $ L $. Minimalita $ A_L $ sa dá dokázať pomocou všeobecne známej Myhill-Nerodeovej vety. Zostrojíme netriviálny rozklad automatu $ A_L $. Hľadané DKA $ A_1^L $ a $ A_2^L $ uvádzame pomocou ich diagramov.

\begin{figure}[H]
\centering
\begin{tikzpicture}[->,>=stealth',shorten >=1pt,auto,node distance=2.5cm,
                    semithick]
   \node[state,initial,accepting] 	(0) 					{$q_0$}; 
   \node[state] 					(1)		[right of=0] 	{$q_1$}; 
   \node[state]						(2)		[right of=1]	{$q_2$};
   \node[state]						(3) 	[right of=2] 	{$q_3$};
   
   \path[->]
    (0) edge [bend right] node {a,b} (1)  
    (1) edge [bend right] node {a,b} (2)
    (2) edge [bend right] node {a,b} (3)
    (3) edge [bend right] node [swap] {a,b} (0)
    ;
\end{tikzpicture}
\begin{tikzpicture}[->,>=stealth',shorten >=1pt,auto,node distance=2.5cm,
                    semithick]
   \node[state,initial,accepting] 	(0) 					{$q_0$}; 
   \node[state] 					(1)		[right of=0] 	{$q_1$}; 
   \node[state]						(T) 	[below right of=0] 	{$q_T$};
   
   \path[->]
    (0) edge [bend right] node {a} 	(1)  
    (1) edge [bend right] node [swap] {a,b} (0)
    (0) edge node [swap] {b} (T)
    (T) edge [loop right] node {a,b} ()
    ;
\end{tikzpicture}
\caption{rozklad automatu $ A_L $}
\end{figure}

Možno nahliadnuť, že jeden z automatov v rozklade počíta zvyšok po delení 4 a druhý kontroluje, či symboly na nepárnych pozíciách v slove sú $ a $. Teda vidno, že $ L(A_1^L) = \lbrace w \in \lbrace a,b \rbrace^* \; | \; |w| \equiv 0 \; (mod \; 4) \rbrace $ a $ L(A_2^L) = (\lbrace a \rbrace \lbrace a,b \rbrace)^* $. Teda $ L(A_1^L) \cap L(A_2^L) = L $. Navyše $ \#_S(A_1^L) < \#_S(A_L) $ a $ \#_S(A_2^L) < \#_S(A_L) $, teda automaty $ A_1^L $ a $ A_2^L $ tvoria netriviálny rozklad automatu $ A_L $. Z predchádzajúceho vyplýva, že jazyk $ L = (\lbrace a \rbrace \lbrace a,b \rbrace \lbrace a \rbrace \lbrace a,b \rbrace)^* $ je deterministicky rozložiteľný. Avšak tento jazyk je podľa Vety \ref{thm:a(a|b)a(a|b)^4} nedeterministicky nerozložiteľný.
\end{proof}

Uvedená Veta síce ukazuje rozdiel medzi deterministickou a nedeterministickou rozložiteľnosťou, avšak jej dôkaz veľmi závisí od faktu, že DKA v definícii nútime k úplnej prechodovej funkcii a vďaka čomu DKA použitý v dôkaze musí mať odpadový stav. Bez tohto odpadového stavu by náš dôkaz neprešiel. Nasledujúca Veta ukazuje, že existujú prípady, kde rozdiel medzi deterministickou a nedeterministickou rozložiteľnosťou nie je spôsobený iba nutnosťou úplnej prechodovej funkcie DKA.

\begin{theorem}
\label{thm:ndet_vs_det_diff_big}
Existuje postupnosť jazykov $ (L_i)_{i=2}^{\infty} $, taká, že platí:
\begin{enumerate}[label=(\alph*)]
\item \label{thm:ndet_vs_det_diff_big_item_1} Jazyk $ L_i $ je nedeterministicky nerozložiteľný a súčasne deterministicky rozložiteľný pre ľubovolné $ i \in \mathbb{N}, i \geq 2 $.
\item \label{thm:ndet_vs_det_diff_big_item_2} Nech pre ľubovolné $ i \in \mathbb{N}, i \geq 3 $ je $ A_i $ minimálny DKA akceptujúci $ L_i $. Potom existuje taký rozklad $ A_i $ na $ A_1^i $ a $ A_2^i $, že platí $ \#_S(A_1^i)=\#_S(A_2^i)=\frac{\#_S(A_i)+3}{2} $.
\end{enumerate}
\end{theorem}

\begin{proof}
Definujme postupnosť jazykov $ P=(L_i)_{i=2}^{\infty} $ nasledovne: $ L_i = (\lbrace a^{i-1} \rbrace \lbrace b \rbrace^* \lbrace a,b \rbrace)^* $ pre ľubovolné $ i \geq 2 $. Spomeňme, že táto postupnosť ešte nie je tá, ktorú hľadáme. Tú, ktorú hľadáme, však dostaneme z $ P $ vybratím niektorých (spočítateľne veľa) jej členov.
\par
Najskôr ukážeme, že pre spočítateľne veľa $ i \geq 2 $ je $ L_i $ nedeterministicky nerozložiteľný. V nasledujúcom uvažujme teda $ i \geq 2 $ také, že $ i $ je mocninou prvočísla. Zostrojíme NKA $ A_i^N $ taký, že $ L(A_i^N) = L_i $. $ A_i^N = (K_i^N, \lbrace a,b \rbrace, \delta_i^N,q_0,\lbrace q_0 \rbrace) $. Kde $ K_i^N = \lbrace q_j \; | \; 0 \leq j < i \rbrace $ a prechodová funkcia $ \delta_i^N $ je definovaná nasledovne - $ \forall 0 \leq j \leq i-2: \delta_i^N(q_j,a) = \lbrace q_{j+1} \rbrace $, $ \delta_i^N(q_{i-1},a)=\lbrace q_0 \rbrace  $, $ \delta_i^N(q_{i-1},b)=\lbrace q_0,q_{i-1} \rbrace  $

Pre ilustráciu a lepšiu čitateľnosť uvádzame automat $ A_4^N $ aj pomocou diagramu.

\begin{figure}[H]
\centering
\begin{tikzpicture}[->,>=stealth',shorten >=1pt,auto,node distance=2.5cm,
                    semithick]
	\node[state,initial,accepting] (0)					{$q_0$};
   	\node[state] 					(1)	[right of=0]	{$q_1$};
   	\node[state] 					(2)	[below of=1]	{$q_2$};
   	\node[state] 					(3)	[left of=2]		{$q_3$};
   	
   	\path[->]
     (0) edge [bend left] node {$ a $} (1)
     (1) edge [bend left] node {$ a $} (2)
     (2) edge [bend left] node {$ a $} (3)
     (3) edge [bend left] node {$ a,b $} (0)
     (3) edge [loop below] node {$ b $} ()
     ;
\end{tikzpicture}
\caption{automat $ A_4^N $}
\end{figure}

Dá sa nahliadnuť, že platí $ L(A_i^N) = L_i $. Navyše, je dobré uvedomiť si, že $ \lbrace a^{ki} \; | \; i \in \mathbb{N} \rbrace \subset L_i $. Uvažujme množinu dvojíc slov $ M_i = \lbrace (a^j,a^{i-j}) \; | \; 0 \leq j < i  \rbrace $. Podľa definície \ref{def:fooling_set} je množina $ M_i $ rozšírenou mätúcou množinou pre jazyk $ L_i $. $ |M_i|=i $, teda podľa Vety \ref{thm:fooling_set_technique} $ nsc(L_i) \geq i $. Kedže $ L(A_i^N) = L_i $ a $ \#_S(A_i^N) = i $, tak $nsc(L_i) = i $ a automat $ A_i^N $ je minimálny NKA pre jazyk $ L_i $.
\par
Ďalej postupujme sporom. Uvažujme, že jazyk $ L_{i} $ je rozložiteľný, teda že existuje netriviálny rozklad automatu $ A_i^N $. To znamená, že existujú NKA $ A_1^{N,i}, A_2^{N,i} $, také, že platí $ \#_S(A_1^{N,i}) < i $, $ \#_S(A_2^{N,i}) < i $, $ L(A_1^{N,i}) \cap L(A_2^{N,i}) = L_i $. Navyše podľa Lemy \ref{lm:nonepsilon_NFA} môžeme predpokladať, že automaty $ A_1^{N,i}$ a $ A_2^{N,i} $ neobsahujú prechody na $ \varepsilon $. 
\par
Nakoľko $ i $ je mocninou prvočísla, tak existuje nejaké prvočíslo $ p $ a nejaké $ n $ také, že $ i=p^n $. Z predchádzajúceho vyplýva, že $ a^{p^n} \in L(A_1^{N,i}), a^{p^n} \in L(A_2^{N,i})$. Teraz sa pozrime na výpočet automatu $ A_1^{N,i} $ na slove $ a^{p^n} $. Nech tento výpočet vyzerá nasledovne $ (q_0,a^{p^n}) \vdash (q_1,a^{p^n-1}) \vdash \dots \vdash (q_{p^n-1},a) \vdash (q_{p^n},\varepsilon) $, kde $ q_0 $ je počiatočný stav automatu $ A_1^{N,i} $, $ q_{p^n} $ je nejaký akceptačný stav automatu $ A_1^{N,i} $ a pre $ 1 \leq j < p^n \; q_j \in K_{A_1^{N,i}}$. Nakolko $ \#_S(A_1^{N,i}) < p^n(=i) $, tak nutne $ \exists j,k \in \mathbb{N}, 0 \leq j,k < p^n, j \neq k: q_j = q_k $ (počas výpočtu sa v časti \glqq{}od začatiatku po predposledný stav\grqq{} nejaký stav zopakuje). Z toho vyplýva, že v akceptovanom slove môžem pumpovať časť, ktorá je kratšia ako $ p^n $, t.j. $ \exists r_1 \in \mathbb{N}, 1 \leq r_1 < p^n \; \forall k \in \mathbb{N}: a^{p^n+kr_1} \in L(A_1^{p^n})$. 
\par
Analogicky, uvažujúc výpočet automatu $ A_2^{N,i} $ na slove $ a^{p^n} $, platí $ \exists r_2 \in \mathbb{N}, 1 \leq r_2 < p^n \; \forall k \in \mathbb{N}: a^{p^n+kr_2} \in L(A_2^{N,i})$.
\par
Čísla $ r_1 $ a $ r_2 $ zapíšme nasledovne. $ r_1 = p^{l_1}f_1, \; 0 \leq l_1 < n, \; p \nmid f_1 $. $ r_2 = p^{l_2}f_2, \; 0 \leq l_2 < n, \; p \nmid f_2 $. Z uvedeného v predošlom vyplýva, že $ a^{p^n + p^{max(l_1,l_2)}f_1f_2} \in L(A_1^{N,i}) \cap L(A_2^{N,i}) $. Nakolko však $ p^n \nmid p^{max(l_1,l_2)}f_1f_2$, tak $ a^{p^n + p^{max(l_1,l_2)}f_1f_2} \notin L_{i} $, čo je však v spore s predpokladom, že automaty $ A_1^{N,i} $ a $ A_2^{N,i} $ tvoria netriviálny rozklad automatu $ A_{i}^N $. Teda jazyk $ L_i $ je nedeterministicky nerozložiteľný.
\par
Ukážeme, že $ L_i $ je deterministicky rozložiteľný pre ľubovoľné $ i \geq 2 $. DKA $ A_i^D $ taký, že $ L(A_i^D)=L_i $, zostrojíme štandardnou podmnožinovou konštrukciou z NKA $ A_i^D $. Takto dostaneme $ A_i^D = (\lbrace q[j],q[j,(j+1) \; mod \; i] \; | \; 0 \leq j < i \rbrace \cup \lbrace q_T \rbrace,\lbrace a,b \rbrace, \delta_i^D, q[0], \lbrace q[0], q[i-1,0],q[0,1] \rbrace) $, kde prechodová funkcia $ \delta_i^D $ je definovaná nasledovne: $ \delta_i^D(q[i-1],b) = q[i-1,0], \; \delta_i^D(q[i-1,0],b) = q[i-1,0], \; \delta_i^D(q[i-2,i-1],b) = q[i-1,0] , \; \forall j \in \mathbb{N}, 0 \leq j < i : \delta_i^D(q[j],a) = q[(j+1) \; mod \; i], \delta_i^D(q[j,(j+1) \; mod \; i],a) = q[(j+1) \; mod \; i, (((j+1) \; mod \; i)+1)  \; mod \; i] $. Naša definícia DKA požaduje úplnosť prechodovej funkcie, teda zatiaľ nedefinované prechody dodefinujeme tak, že idú automaticky do odpadového stavu $ q_T $. Minimalita $ A_i^D $ sa dá dokázať pomocou všeobecne známej Myhill-Nerodeovej vety. Pre ilustráciu a lepšiu čitateľnosť uvádzame automat $ A_4^D $ aj pomocou diagramu. V diagrame pre prehľadnosť neuvádzame odpadový stav $ q_T $.

\begin{figure}[H]
\centering
\begin{tikzpicture}[->,>=stealth',shorten >=1pt,auto,node distance=2.5cm,
                    semithick]
	\node[state,initial,accepting] (0)						{$q[0]$};
   	\node[state] 					(1)		[below of=0]	{$q[1]$};
   	\node[state] 					(2)		[right of=1]	{$q[2]$};
   	\node[state] 					(3)		[above of=2]	{$q[3]$};
	\node[state,accepting] 			(30)	[right of=3]	{$q[3,0]$};
   	\node[state,accepting]			(01)	[right of=30]	{$q[0,1]$};
   	\node[state] 					(12)	[below of=01]	{$q[1,2]$};
   	\node[state] 					(23)	[left of=12]	{$q[2,3]$};
   	
   	\path[->]
     (0) edge [bend right] node [swap] {$ a $} (1)
     (1) edge [bend right] node [swap] {$ a $} (2)
     (2) edge [bend right] node [swap] {$ a $} (3)
     (3) edge [bend right] node [swap] {$ a $} (0)
     (3) edge [bend left] node {$ b $} (30)
     (30) edge [bend left] node {$ a $} (01)
     (01) edge [bend left] node {$ a $} (12)
     (12) edge [bend left] node {$ a $} (23)
     (23) edge [bend left] node {$ a,b $} (30)
     (30) edge [loop above] node {$ b $} ()
     ;
\end{tikzpicture}
\caption{automat $ A_4^D $}
\end{figure}

Zostrojíme netriviálny rozklad DKA $ A_i^D $. Myšlienkou rozkladu je, neformálne povedané, že v jednotlivých automatoch rozkladu budeme mať namiesto oboch úplných cyklov, ktoré sú v $ A_i^D $ jeden cyklus \glqq{}spľasnutý\grqq{} do jedného stavu a druhý cyklus úplný. Keď budeme uvažovať slová, ktoré budú akceptované oboma automatmi, tak vždy jeden automat správne zráta daný cyklus, čo nám bude stačiť. Uvedomme si ešte, že rozklad nám bude správne fungovať aj vďaka faktu, že dané dva cykly sú oddelené práve jedným prechodom na $ b $ z prvého cyklu do druhého, pričom v prvom cykle prechody na $ b $ nepoužívame. Formálne definujeme automaty $ A_1^{D,i}, A_2^{D,i} $. 
\begin{enumerate}
\item $ A_1^{D,i} = (\lbrace q_{C1}, q_T \rbrace \cup \lbrace q[j,(j+1) \; mod \; i] \; | \; 0 \leq j < i \rbrace, \lbrace a,b \rbrace, \delta_1^{D,i}, q_{C1}, F_1^{D,i}) $, kde $ F_1^{D,i} = \lbrace q_{C1}, q[i-1,0],q[0,1] \rbrace$ a prechodová funkcia $ \delta_1^{D,i} $ je definovaná nasledovne: $ \delta_1^{D,i}(q_{C1},a) = q_{C1}, \; \delta_1^{D,i}(q_{C1},b) = q[i-1,0], \; \delta_i^D(q[i-1,0],b) = q[i-1,0], \; \delta_i^D(q[i-2,i-1],b) = q[i-1,0], \; \forall j \in \mathbb{N}, 0 \leq j < i : \delta_i^D(q[j,(j+1) \; mod \; i],a) = q[(j+1) \; mod \; i, (((j+1) \; mod \; i)+1)  \; mod \; i] $. Prechody, ktoré sme zatiaľ nedefinovali, dodefinujeme tak, že automaticky vedú do odpadového stavu $ q_T $ 
\item $ A_2^{D,i} = (\lbrace q_{C2}, q_T \rbrace \cup \lbrace q[j] \; | \; 0 \leq j < i \rbrace, \lbrace a,b \rbrace, \delta_2^{D,i}, q[0], F_2^{D,i}) $, kde $ F_2^{D,i} = \lbrace q[0], q_{C2} \rbrace $ a prechodová funkcia $ \delta_2^{D,i} $ je definovaná nasledovne: $ \delta_2^{D,i}(q[i-1],b) = q_{C2}, \; \delta_1^{D,i}(q_{C2},a) = q_{C2}, \; \delta_1^{D,i}(q_{C2},b) = q_{C2}, \forall j \in \mathbb{N}, 0 \leq j < i : \delta_i^D(q[j],a) = q[(j+1) \; mod \; i] $. Prechody, ktoré sme zatiaľ nedefinovali, dodefinujeme tak, že automaticky vedú do odpadového stavu $ q_T $
\end{enumerate}
Pre ilustráciu a lepšiu čitateľnosť uvádzame rozklad automatu $ A_4^D $ na automaty $ A_1^{D,4}, A_2^{D,4} $ aj pomocou diagramu. V diagrame pre prehľadnosť neuvádzame odpadový stav $ q_T $.

\begin{figure}[H]
\centering
\begin{tikzpicture}[->,>=stealth',shorten >=1pt,auto,node distance=2.5cm,
                    semithick]
	\node[state,initial,accepting] (C1)					{$q_{C1}$};
	\node[state,accepting] 			(30)	[right of=C1]	{$q[3,0]$};
   	\node[state,accepting]			(01)	[above right of=30]	{$q[0,1]$};
   	\node[state] 					(12)	[below right of=01]	{$q[1,2]$};
   	\node[state] 					(23)	[below left of=12]	{$q[2,3]$};
   	
   	\path[->]
     (C1) edge [loop above] node {$ a $} ()
     (C1) edge [bend left] node {$ b $} (30)
     (30) edge [bend left] node {$ a $} (01)
     (01) edge [bend left] node {$ a $} (12)
     (12) edge [bend left] node {$ a $} (23)
     (23) edge [bend left] node {$ a,b $} (30)
     (30) edge [loop right] node {$ b $} ()
     ;
\end{tikzpicture}

\begin{tikzpicture}[->,>=stealth',shorten >=1pt,auto,node distance=2.5cm,
                    semithick]
	\node[state,initial,accepting] (0)							{$q[0]$};
   	\node[state] 					(1)		[below left of=0]	{$q[1]$};
   	\node[state] 					(2)		[below right of=1]	{$q[2]$};
   	\node[state] 					(3)		[above right of=2]	{$q[3]$};
	\node[state,accepting] 			(C2)	[right of=3]		{$q_{C2}$};
   	
   	\path[->]
     (0) edge [bend right] node [swap] {$ a $} (1)
     (1) edge [bend right] node [swap] {$ a $} (2)
     (2) edge [bend right] node [swap] {$ a $} (3)
     (3) edge [bend right] node [swap] {$ a $} (0)
     (3) edge [bend left] node {$ b $} (C2)
     (C2) edge [loop above] node {$ a,b $} ()
     ;
\end{tikzpicture}
\caption{rozklad automatu $ A_4^D $ na automaty $ A_1^{D,4}$(hore) a $ A_2^{D,4} $(dole)}
\end{figure}

Ukážeme, že $ L(A_i^D) = L(A_1^{D,i}) \cap L(A_2^{D,i}) $. \\
$ \subseteq: $ Ľahko vidno z konštrukcie automatov $ A_1^{D,i}$ a $ A_2^{D,i} $. \\
$ \supseteq: $ Uvažujme $ w \in L(A_1^{D,i}) \cap L(A_2^{D,i}) $. Teda existujú stavy $ q_{F1} \in F_1^{D,i}, q_{F2} \in F_2^{D,i}$ také, že $ (q_{C1},w) \vdash_{A_1^{D,i}}^* (q_{F1}, \varepsilon), (q[0],w) \vdash_{A_2^{D,i}}^* (q_{F2}, \varepsilon)$. Rozoberieme postupne, aký môže byť stav $ q_{F1} $.
\begin{enumerate}
\item $ q_{F1} = q_{C1} $. Z konštrukcie $ A_1^{D,i} $ plynie, že výpočet $ (q_{C1},w) \vdash_{A_1^{D,i}}^* (q_{F1}, \varepsilon)$ prechádza iba cez stav $ q_{C1} $. Teda existuje nejaké $ n \in \mathbb{N} $ také, že $ w=a^n $. Z konštrukcie $ A_2^{D,i} $ teda $ q_{F2} = q[0] $ a výpočet $ (q[0],w) \vdash_{A_2^{D,i}}^* (q_{F2}, \varepsilon) $ je zároveň akceptačným výpočtom automatu $ A_i^D $ na slove $ w $. Neformálne, automat $ A_i^D $ používa iba prvý svoj cyklus, ktorý je ale celý obsiahnutý aj v $ A_2^{D,i} $.
\item $ q_{F1} \in \lbrace q[i-1,0],q[0,1] \rbrace $. Potom z konštrukcie $ A_1^{D,i} $ vyplýva, že existujú nejaké $ n \in \mathbb{N}, u \in \lbrace a,b \rbrace^* $ také, že $ w = a^nbu $. Výpočet automatu $ A_1^{D,i} $ na slove $ w $ teda vyzerá nasledovne: $ (q_{C1},a^nbu) \vdash_{A_1^{D,i}}^* (q_{C1},bu) \vdash_{A_1^{D,i}} (q[i-1,0],u) \vdash_{A_1^{D,i}}^* (q_{F1, \varepsilon}) $. Nakoľko slovo $ w $ obsahuje symbol $ b $, tak z konštrukcie $ A_2^{D,i} $ vyplýva $ q_{F2} = q_{C2} $ a výpočet automatu $ A_1^{D,i} $ na slove $ w $ vyzerá nasledovne: $ (q[0],a^nbu) \vdash_{A_2^{D,i}}^* (q[i-1],bu) \vdash_{A_2^{D,i}} (q_{C2},u) \vdash_{A_2^{D,i}}^* (q_{C2},\varepsilon)$. Z konštrukcie $ A_1^{D,i} $ plynie, že výpočet $ (q[i-1,0],u) \vdash_{A_1^{D,i}}^* (q_{F1}, \varepsilon) $ v automate $ A_1^{D,i} $ je zároveň výpočtom $ (q[i-1,0],u) \vdash_{A_i^D}^* (q_{F1}, \varepsilon) $ v automate $  A_i^D $. Z konštrukcie $ A_2^{D,i} $ plynie, že výpočet $ (q[0],a^nbu) \vdash_{A_2^{D,i}}^* (q[i-1],bu) $ v automate $ A_2^{D,i} $ je zároveň výpočtom $ (q[0],a^nbu) \vdash_{A_i^D}^* (q[i-1],bu) $ v automate $  A_i^D $. Navyše platí $ \delta_i^D(q[i-1],b) = q[i-1,0] $. Z toho vyplýva $ (q[0],a^nbu) \vdash_{A_i^D}^* (q[i-1],bu) \vdash_{A_i^D} (q[i-1,0],u) \vdash_{A_i^D}^* (q_{F1, \varepsilon}) $. Nakoľko $ q_{F1} \in F_i^D $, tak tento výpočet je akceptačným výpočtom automatu $ A_i^D $ na slove $ w $. Neformálne povedané, oba z automatov v rozklade zrátajú jeden cyklus z pôvodného automatu a v tom druhom len stoja. V prieniku dostaneme teda zrátané oba cykly.
\end{enumerate}

Nakoľko iné možnosti neexistujú, tak platí $ L(A_i^D) = L(A_1^{D,i}) \cap L(A_2^{D,i}) $. Zjavne $ \#_S(A_1^{D,i}) < \#_S(A_i^D), \#_S(A_2^{D,i}) < \#_S(A_i^D) $ a teda automaty $ A_1^{D,i} $ a $ A_2^{D,i} $ tvoria netriviálny rozklad automatu $ A_i^D $. Teda $ L_i $ je deterministicky rozložiteľný pre ľubovoľné $ i \geq 2 $.
\par
Teraz zhrňme, čo sme dokázali. Pre postupnosť jazykov $ P=(L_i)_{i=2}^{\infty} $ platí, že obsahuje nekonečne veľa jazykov, ktoré sú súčasne nedeterministicky nerozložiteľné a deterministicky rozložiteľné. Sú to tie $ L_i $ pre ktoré je $ i $ mocninou prvočísla. Hľadanú postupnosť teda získame tak, že z postupnosti $ P $ vytvoríme novú postupnosť $ Q $ tak, že z $ P $ vyberieme tie $ L_i $, kde $ i $ je mocninou prvočísla. Zjavne postupnosť $ Q $ spĺňa \ref{thm:ndet_vs_det_diff_big_item_1} a pri lepšom pohľade na dôkaz deterministickej rozložiteľnosti jazykov $ L_i $ spĺňa aj \ref{thm:ndet_vs_det_diff_big_item_2}. Teda $ Q $ je hľadaná postupnosť.

\end{proof}

\section{Automaty tvorené jediným cyklom}

Typickou schopnosťou konečných automatov je počítať v cykle zvyšok po delení dĺžky slova. Tieto automaty sa vyznačujú tým, že sú tvorené jediným cyklom, pričom nijak nezohľadňujú štruktúru slova. Podstatu otázok spojených s takýmito automatmi riešia Vety \ref{thm:prime^n} a \ref{thm:compound}. Nakoľko v konečných automatoch sú práve cykly veľmi dôležitou štruktúrou, v našej práci sme túto otázku rozšírili a študovali sme otázku rozložiteľnosti jazykov, ktorých minimálne nedeterministické konečné automaty sú tvorené jediným cyklom, pričom v ňom zohľadňujú aj štruktrúru akceptovaného slova. Podstatou týchto automatov je, neformálne povedané, pumpovanie nejakého slova.
\par
Pre lepšiu čitateľnosť dôkazov zavedieme nasledujúce označenia.

\begin{notation}
\normalfont
Nech $ u $ je ľubovolné slovo, $ k \in \mathbb{N} $. Potom $ pref(u,k) $ označujeme prefix slova $ u $ dĺžky $ k $ a $ suff(u,k) $ označujeme suffix slova $ u $ dĺžky $ k $.
\end{notation}

\begin{notation}
\normalfont
Nech $ u = u_1u_2 \ldots u_n$ je ľubovolné slovo. Ak v diagrame NKA $ A $ použijeme nasledujúce označenie:
\begin{figure}[H]
\centering
\begin{tikzpicture}[->,>=stealth',shorten >=1pt,auto,node distance=2.5cm,
                    semithick]
   \node[state] 	(1)		 				{$q$}; 
   \node[state]		(2)		[right of=1]	{$p$};
   
   \path[->]
    (1) edge node {u} (2)  
    ;
\end{tikzpicture}
\end{figure}

Myslíme tým, že v automate $ A $ sa dá zo stavu $ q $ dostať do stavu $ p $ na slovo $ u $ pričom zo stavov, v ktorých sa automat $ A $ nachádza počas čítania slova $ u $ sa nedá už nikam inam dostať. Formálne existujú $ q_0, q_1, \ldots, q_n \in K_A $ také, že $ q_0 = q, q_n = p, \delta_A(q, u_1) \ni q_1 $ a pre $ 0 < i < n $ platí $ \delta_A(q_i,u_{i+1}) = \lbrace q_{i+1} \rbrace, q_i \notin F_A $. Treba si uvedomiť, že pokiaľ $ u = \varepsilon $, tak platí $ q=p $, ak navyše v tom prípade aspoň jeden zo stavov je označený v diagrame ako akceptačný, tak sa tým myslí, že stav je akceptačný.
\end{notation}

\begin{lemma}
\label{lm:one_word_cycle_nsc}
Nech $ \Sigma $ je ľubovolná abeceda, nech $ u \in \Sigma^* $, nech $ L_u = \lbrace u \rbrace^* $. Potom $ nsc(L_u) = |u| $.
\end{lemma}

\begin{proof}
Zostrojíme NKA $ A_u $ pre jazyk $ L_u $. Automat uvádzame pomocou diagramu.

\begin{figure}[H]
\centering
\begin{tikzpicture}[->,>=stealth',shorten >=1pt,auto,node distance=2cm,
                    semithick]
   \node[state,initial,accepting] 	(0) 					{$q_0$}; 
   
   \path[->]
    (0) edge [loop right] node {u} ()  
    ;
\end{tikzpicture}
\caption{automat $ A_u $}
\end{figure}
Ľahko vidno, že $ L(A_u) = L_u $. Uvažujme množinu dvojíc slov $ F = \lbrace (pref(u,i), \; suff(|u|-i)) \; | \; 0 \leq i < |u| \rbrace $. Množina $ F $ je podľa definície \ref{def:fooling_set} oblbovacou množinou pre jazyk $ L_u $. Nakoľko $ |F|=|u| $, tak podľa Vety \ref{thm:fooling_set_technique} $ nsc(L_u) \geq |u| $. Kedže $ L(A_u) = L_u $ a $ \#_S(A_u) = |u| $, tak $ nsc(L_u) = |u| $ a automat $ A_u $ je minimálny NKA pre jazyk $ L_u $.
\end{proof}

\begin{theorem}
Nech $ \Sigma $ je ľubovolná abeceda taká, že $ |\Sigma| \geq 2 $. Nech pre $ u \in \Sigma^* $, $ k \geq 2 $ je $ L_u^k = \lbrace u^k \rbrace^* $. Ak $ u $ obsahuje aspoň dva rôzne symboly, potom je $ L_u^k $ rozložiteľný.
\end{theorem}

\begin{proof}
Nech $ n \geq 1, \Sigma = \lbrace a, b_1, \ldots , b_n \rbrace, u \in \Sigma^*,$ $ u $ obsahuje symbol $ a $ a minimálne ešte jeden symbol zo $ \Sigma $. Podľa Lemy \ref{lm:one_word_cycle_nsc} platí $ nsc(L_u^k) = k|u| $. Teda existuje NKA $ A_u^k $ taký, že $ L(A_u^k) = L_u^k $ a $ \#_S(A_u^k) = k|u| $. Automat $ A_u^k $ je teda minimálny NKA pre $ L_u^k $. Zostrojíme netriviálny rozklad automatu $ A_u^k $. Označme $ l = k.\#_a(u) $. Rozklad uvádzame pomocou diagramu.

\begin{figure}[H]
\centering
\begin{tikzpicture}[->,>=stealth',shorten >=1pt,auto,node distance=2cm,
                    semithick]
   \node[state,initial,accepting] 	(0) 					{$q_0$}; 
   
   \path[->]
    (0) edge [loop above] node [swap] {u} ()  
    ;
\end{tikzpicture}

\begin{tikzpicture}[->,>=stealth',shorten >=1pt,auto,node distance=2.21cm,
                    semithick]
   \node[state,initial] 	(0) 					{$[0]$}; 
   \node[state] 			(1)		[right of=0] 	{$[1]$}; 
   \node[state] 			(2)		[right of=1] 	{$[2]$};
   \node					(dots)	[right of=2]	{$\ldots$};
   \node[state]				(auk_1) 	[right of=dots] 	{$[l-1]$};
   \node[state,accepting]	(auk) 	[right of=auk_1] 	{$[l]$};
   
   \path[->]
    (0) edge [bend left] node {a} 	(1)
    (1) edge [bend left] node {a} 	(2)
    (2) edge [bend left] node {a} 	(dots)
    (dots) edge [bend left] node {a} 	(auk_1)
    (auk_1) edge [bend left] node {a} 	(auk)
    (auk) edge [bend left = 35] node {a} 	(0)
   	(0) edge [loop above] node {$ b_1, \ldots, b_n $} ()
   	(1) edge [loop above] node {$ b_1, \ldots, b_n $} ()
   	(2) edge [loop above] node {$ b_1, \ldots, b_n $} ()
   	(auk_1) edge [loop above] node {$ b_1, \ldots, b_n $} ()
 	(auk) edge [loop above] node {$ b_1, \ldots, b_n $} ()
    ;
\end{tikzpicture}

\caption{rozklad automatu $ A_u^k $ na automaty $ A_u $(hore) a $ A_k $(dole)}
\end{figure}

Myšlienkou tohto rozkladu je, že jeden z automatov kontroluje štruktúru slova, či je práve niekoľkonásobným zreťazením slova $ u $ a druhý automat kontroluje, či je slov $ u $ správne veľa. To však robí tak, že iba počíta počet nejakého jedného symbolu (v našom prípade ho označujeme $ a $), ktorý $ u $ obsahuje, pričom kontroluje, či slovo obsahuje práve $ m.k.\#_a(u) $ pre nejaké $ m \in \mathbb{N} $. Formálne $ L(A_u) = \lbrace u \rbrace^* $ a $ L(A_k) = \lbrace w \in \Sigma^* \; | \; \#_a(w) \equiv 0 \; (mod \; k.\#_a(u)) \rbrace $. Teda $ L(A_u) \cap L(A_k) = L(A_u^k) $. Navyše $ \#_S(A_u) < \#_S(A_u^k) $ a $ \#_S(A_k) < \#_S(A_u^k) $. Je dobré si uvedomiť, že kvôli prvej nerovnosti potrebujeme predpoklad $ k \geq 2 $ a kvôli druhej nerovnosti potrebujeme predpoklad o veľkosti abecedy $ \Sigma $. Teda automaty $ A_u $ a $ A_k $ tvoria netriviálny rozklad automatu $ A_u^k $.

\end{proof}

\begin{theorem}
Nech $ \Sigma $ je ľubovolná abeceda, nech $ k_1,k_2 \in \lbrace 0,1 \rbrace $, nech $ w_1,w_2,w_3,w_4,w_5,w_6 \in \Sigma^* $. Definujeme $ L = \lbrace w_1a^{k_1}w_2bw_3aw_4bw_5a^{k_2}w_6 \rbrace^* $. Ak $ k_1 = 1 $ alebo $ k_2 = 1 $, potom je $ L $ rozložiteľný.
\end{theorem}

\begin{proof}
Zaveďme označenia $ u = w_1a^{k_1}w_2bw_3aw_4bw_5a^{k_2}w_6$ a $ \Sigma_{ab} = \Sigma \cup \lbrace a,b \rbrace $. Podľa Lemy \ref{lm:one_word_cycle_nsc} platí $ nsc(L) = |u| $. Teda existuje NKA $ A $ taký, že $ L(A) = L $ s $ \#_S(A) = |u| $. Automat $ A $ je teda minimálny NKA pre $ L $. Zostrojíme netriviálny rozklad automatu $ A $. Rozoberieme nasledujúce dva prípady, podľa toho akého tvaru je slovo $ u $. Podľa predpokladov je $ u $ práve jedého z nasledujúcich tvarov:

\begin{enumerate}
\item Existujú dve rôzne podslová v slove $ u $ také, že symbol $ b $ je nasledovaný symbolom rôznym od $ b $. Formálne existujú $ v_1,v_2,v_3 \in \Sigma_{ab}^* $ a $ \overline{b}_1,\overline{b}_2 \in \Sigma_{ab} - \lbrace b \rbrace $ také, že $ u = v_1b\overline{b}_1v_2\overline{b}_2v_3 $. Na základe tohto poznatku zostrojíme netriviálny rozklad automatu $ A $. Rozklad uvádzame pomocou diagramu.

\begin{figure}[H]
\centering
\begin{tikzpicture}[->,>=stealth',shorten >=1pt,auto,node distance=2cm,
                    semithick]
	\node[state,initial,accepting] (0)							{$q_0$};
   	\node[state] 					(v1)	[right of=0]		{$q_{v_1}^{b}$};
   	\node[state] 					(notb1)	[right of=v1]		{$q_{\overline{b}_1}$};      	\node[state] 					(v2)	[right of=notb1]	{$q_{v_2}$};
   	\node[state] 					(b)		[right of=v2]		{$q_b$};
   	\node[state] 					(notb2)	[right of=b]		{$q_{\overline{b}_2}$};
   
   	\path[->]
     (0) edge [bend left] node {$ v_1 $} (v1) 
     (v1) edge [bend left] node {$ \overline{b}_1 $} (notb1)
     (v1) edge [loop above] node {$ b $} ()
     (notb1) edge [bend left] node {$ v_2 $} (v2)
     (v2) edge [bend left] node {$ b $} (b)
     (b) edge [bend left] node {$ \overline{b}_2 $} (notb2)
     (notb2) edge [bend left] node {$ v_3 $} (0)
     ;
\end{tikzpicture}

\begin{tikzpicture}[->,>=stealth',shorten >=1pt,auto,node distance=2cm,
                    semithick]
	\node[state,initial,accepting] (0)							{$q_0$};
   	\node[state] 					(v1)	[right of=0]		{$q_{v_1}$};
 	\node[state] 					(b)		[right of=v1]		{$q_b$};
   	\node[state] 					(notb1)	[right of=b]		{$q_{\overline{b}_1}$}; 	
	\node[state] 					(v2)	[right of=notb1]	{$q_{v_2}^b$};
   	\node[state] 					(notb2)	[right of=v2]		{$q_{\overline{b}_2}$};
   
   	\path[->]
     (0) edge [bend left] node {$ v_1 $} (v1) 
     (v1) edge [bend left] node {$ b $} (b)
     (b) edge [bend left] node {$ \overline{b}_1 $} (notb1)
     (notb1) edge [bend left] node {$ v_2 $} (v2)
     (v2) edge [loop above] node {$ b $} ()
     (v2) edge [bend left] node {$ \overline{b}_2 $} (notb2)
     (notb2) edge [bend left] node {$ v_3 $} (0)
     ;
\end{tikzpicture}
\caption{rozklad automatu $ A $ na automaty $ A_1 $ a $ A_2 $}
\end{figure}

Možno nahliadnuť, že $ L(A_1) = \lbrace v_1b^{l}\overline{b}_1v_2b\overline{b}_2v_3 \; | \; l \in \mathbb{N} \rbrace^* $ a $ L(A_2) = \lbrace v_1b\overline{b}_1v_2b^{l}\overline{b}_2v_3 \; | \; l \in \mathbb{N} \rbrace^* $.
\par 
Dokážeme, že $ L(A_1) \cap L(A_2) = L$. \\
$ \supseteq: $ Táto inklúzia je triviálna, nebudeme ju formálne dokazovať. \\
$ \subseteq: $ Uvažujme $ w \in L(A_1) \cap L(A_2) $. Potom existuje $ n,m,l_1,\ldots,l_n,o_1,\ldots,o_m \in \mathbb{N} $ také, že $ w = v_1b^{l_1}\overline{b}_1v_2b\overline{b}_2v_3 \ldots v_1b^{l_n}\overline{b}_1v_2b\overline{b}_2v_3 = v_1b\overline{b}_1v_2b^{o_1}\overline{b}_2v_3 \ldots v_1b\overline{b}_1v_2b^{o_m}\overline{b}_2v_3 $. Indukciou na $ n $ dokážeme, že $ m=n $, pre $ 0 \leq i \leq n: l_i=1 $ a pre $ 0 \leq i \leq m: o_i=1 $.

\begin{itemize}
    \item [$ 1^0: $] Ak $ n=0 $, tak $ w = \varepsilon $ a tvrdenie triviálne platí.
    \item [$ 2^0: $] Platí $ v_1b^{l_1}\overline{b}_1v_2b\overline{b}_2v_3 \ldots v_1b^{l_n}\overline{b}_1v_2b\overline{b}_2v_3 = v_1b\overline{b}_1v_2b^{o_1}\overline{b}_2v_3 \ldots v_1b\overline{b}_1v_2b^{o_m}\overline{b}_2v_3 $. Pozrime sa pozornejšie na prvé úseky v tomto slove, t.j. na časti $ v_1b^{l_1}\overline{b}_1v_2b\overline{b}_2v_3 $ a $ v_1b\overline{b}_1v_2b^{o_1}\overline{b}_2v_3 $. Oba úseky sú prefixom toho istého slova a na prvých $ |v_1| $ symboloch sa evidentne zhodujú. Musí platiť $ l_1 \geq 1 $, aby sa zhodovali aj na symoble $ b $, ktorý nasleduje za $ v_1 $. Avšak nakoľko v tomto prefixe po zmienenom $ b $ nasleduje znak $ \overline{b}_1 $, tak nutne $ l_1=1 $. Teda platí $ v_1b^{l_1}\overline{b}_1v_2 = v_1b\overline{b}_1v_2 $. Z toho plynie $ o_1 \geq 1 $, nakoľko po $ v_2 $ musí nasledovať symbol $ b $. Ďalším sybmobolom je však $ \overline{b}_2 $, teda nutne $ o_1 = 1 $. Teda platí $ v_1b^{l_1}\overline{b}_1v_2b\overline{b}_2v_3 = v_1b\overline{b}_1v_2b^{o_1}\overline{b}_2v_3 = v_1b\overline{b}_1v_2b\overline{b}_2v_3$. Navyše, oba automaty, $ A_1 $ aj $ A_2 $ sa po dočítaní tohto prefixu dostanú práve do ich počiatočného (a zároveň jediného akceptačného) stavu $ q_0 $. V prípade, že $ n=1 $, tak niet čo ďalej dokazovať. Ak $ n \geq 2 $ tak z predchádzajúceho vyplýva, že $ v_1b^{l_2}\overline{b}_1v_2b\overline{b}_2v_3 \ldots v_1b^{l_n}\overline{b}_1v_2b\overline{b}_2v_3 = v_1b\overline{b}_1v_2b^{o_2}\overline{b}_2v_3 \ldots v_1b\overline{b}_1v_2b^{o_m}\overline{b}_2v_3 $ a navyše toto slovo akceptujú oba automaty, $ A_1 $ aj $ A_2 $. Teda podľa indukčného predpokladu môžeme tvrdiť, že $ n=m $, pre $ 2 \leq i \leq n $ platí $ l_i = o_i = 1 $, čo dokazuje tvrdenie.
\end{itemize}
Z predošlého vyplýva, že $ w \in L $, čo kompletizuje dôkaz tejto inklúzie. 
\par
Teda $ L(A_1) \cap L(A_2) = L = L(A) $. Navyše $ \#_S(A_1) < \#_S(A) $ a $ \#_S(A_2) < \#_S(A) $, teda automaty $ A_1 $ a $ A_2 $ tvoria netriviálny rozklad automatu $ A $.

\item Existujú $ \overline{b}_1,\overline{b}_2 \in \Sigma_{ab} - \lbrace b \rbrace $, $ v_1,v_2 \in \Sigma_{ab} - \lbrace b \rbrace $, $ c_1,c_2 \geq 1 $ také, že $ u = \overline{b}_1v_1b^{c_1}\overline{b}_2v_2b^{c_2} $. Na základe tohto poznatku zostrojíme netriviálny rozklad automatu $ A $. Rozklad uvádzame pomocou diagramu.

\begin{figure}[H]
\centering
\begin{tikzpicture}[->,>=stealth',shorten >=1pt,auto,node distance=2cm,
                    semithick]
	\node[state,initial,accepting] (0)							{$q_0$};
   	\node[state] 					(notb1)		[right of=0]	{$q_{\overline{b}_1}$};
   	\node[state] 					(v1)		[right of=notb1]	{$q_{v_1}^b$};
   	\node[state] 					(notb2)		[right of=v1]	{$q_{\overline{b}_2}$};
   	\node[state] 					(v2)		[right of=notb2] {$q_{v_2}$};
   
   	\path[->]
     (0) edge [bend left] node {$ \overline{b}_1 $} (notb1)
     (notb1) edge [bend left] node {$ v_1 $} (v1)
     (v1) edge [loop above] node {$ b $} ()
     (v1) edge [bend left] node {$ \overline{b}_2 $} (notb2)
     (notb2) edge [bend left] node {$ v_2 $} (v2)
     (v2) edge [bend left] node {$ b $} (0)
     ;
\end{tikzpicture}
\begin{tikzpicture}[->,>=stealth',shorten >=1pt,auto,node distance=2cm,
                    semithick]
	\node[state,initial,accepting] (0)							{$q_0^b$};
   	\node[state] 					(notb1)		[right of=0]	{$q_{\overline{b}_1}$};
   	\node[state] 					(v1)		[right of=notb1]	{$q_{v_1}$};
   	\node[state] 					(b)			[right of=v1]	{$q_{b}$};
   	\node[state] 					(notb2)		[right of=b]	{$q_{\overline{b}_2}$};
   
   	\path[->]
     (0) edge [bend left] node {$ \overline{b}_1 $} (notb1)
     (notb1) edge [bend left] node {$ v_1 $} (v1)
     (v1) edge [bend left] node {$ b $} (b)
     (b) edge [bend left] node {$ \overline{b}_2 $} (notb2)
     (notb2) edge [bend left] node {$ v_2 $} (0)
     (0) edge [loop above] node {$ b $} ()
     ;
\end{tikzpicture}
\caption{rozklad automatu $ A $ na automaty $ A_1 $(hore) a $ A_2 $(dole)}
\end{figure}

Možno nahliadnuť, že $ L(A_1) = \lbrace \overline{b}_1v_1b^l\overline{b}_2v_2b \; | \; l \in \mathbb{N} \rbrace^* $ a $ L(A_2) = \lbrace b^l\overline{b}_1v_1b\overline{b}_2v_2 \; | \; l \in \mathbb{N} \rbrace^*\lbrace b \rbrace^* $. Platí $ L(A_1) \cap L(A_2) = L $, čo sa dá dokázať veľmi podobne a rovnako veľmi technicky ako v predošlom prípade, preto dôkaz neuvádzame. Navyše $ \#_S(A_1) < \#_S(A) $ a $ \#_S(A_2) < \#_S(A) $, teda automaty $ A_1 $ a $ A_2 $ tvoria netriviálny rozklad automatu $ A $.
\end{enumerate}

Záverom ešte spomeňme, že hlavnou myšlienkou rozkladu bola akási synchronizácia výpočtov automatov v rozklade na symboloch rôznych od $ b $, ktoré nasledovali hneď za $ b $.
\end{proof}

\section{Charakterizácia jazykov tvorených jedným slovom}
Uvádzame úplnú charakterizáciu triedy jazykov tvorených práve jedným slovom vzhľadom na rozložiteľnosť.

\begin{theorem}
\label{thm:singleton_characterization}
Nech $ L = \lbrace w \rbrace $. Potom je $ L $ rozložiteľný práve vtedy, keď $ w $ obsahuje aspoň dva rôzne symboly.
\end{theorem}

\begin{proof}
$ \Rightarrow $: Dokážeme obmenu tvrdenia. Ak $ w $ obsahuje nanajvýš jeden znak, tak existuje nejaké $ n \in \mathbb{N} $ také, že $ L = \lbrace a^n \rbrace $. Potom podľa Vety \ref{thm:Sigma^n} je jazyk $ L $ nerozložiteľný.
$ \Leftarrow $: Nech $ w = w_1abw_2 $ pre nejaké slová $ w_1,w_2 $. Zostrojíme NKA $ A_w $ pre jazyk $ L = \lbrace w \rbrace $. Automat uvádzame pomocou diagramu.

\begin{figure}[H]
\centering
\begin{tikzpicture}[->,>=stealth',shorten >=1pt,auto,node distance=2.5cm,
                    semithick]
   \node[state,initial] 			(0) 					{$q_0$}; 
   \node[state] 					(w1)	[right of=0] 	{$q_{w_1}$}; 
   \node[state]						(a)		[right of=w1]	{$q_{a}$};
   \node[state]						(b) 	[right of=a] 	{$q_{b}$};
   \node[state,accepting]			(w2) 	[right of=b] 	{$q_{w_2}$};
   
   \path[->]
    (0) edge [bend left] node {$ w_1 $} (w1)  
    (w1) edge [bend left] node {a} (a)
    (a) edge [bend left] node {b} (b)
    (b) edge [bend left] node {$ w_2 $} (w2)
    ;
\end{tikzpicture}
\caption{automat $ A_w $}
\end{figure}

Uvažujme množinu dvojíc slov $ F = \lbrace (pref(w,i),suff(w,|w|-i)) \; | \; 0 \leq i \leq |u| \rbrace $. Množina $ F $ je podľa definície \ref{def:fooling_set} oblbovacou množinou pre jazyk $ L $. Nakoľko $ |F| = |w|+1 $, tak podľa Vety \ref{thm:fooling_set_technique} $ nsc(L) \geq |w|+1 $. Kedže $ L(A_w) = L $ a $ \#_S(A_w) = |w|+1 $, tak $ nsc(L) = |w|+1 $ a automat $ A_w $ je minimálny NKA pre jazyk $ L $.
\par
Zostrojíme netriviálny rozklad automatu $ A_w $. Hľadané automaty $ A^a_w $ a $ A^b_w $ uvádzame pomocou diagramov.

\begin{figure}[H]
\centering
\begin{tikzpicture}[->,>=stealth',shorten >=1pt,auto,node distance=2.5cm,
                    semithick]
   \node[state,initial] 			(0) 					{$q_0$}; 
   \node[state] 					(w1a)	[right of=0] 	{$q_{w_1a}$}; 
   \node[state]						(b) 	[right of=w1a] 	{$q_{b}$};
   \node[state,accepting]			(w2) 	[right of=b] 	{$q_{w_2}$};
   
   \path[->]
    (0) edge [bend left] node {$ w_1 $} (w1a)  
    (w1a) edge [bend left] node {b} (b)
    (b) edge [bend left] node {$ w_2 $} (w2)
    (w1a) edge [loop above] node {a} ()
    ;
\end{tikzpicture}

\begin{tikzpicture}[->,>=stealth',shorten >=1pt,auto,node distance=2.5cm,
                    semithick]
   \node[state,initial] 			(0) 					{$q_0$}; 
   \node[state] 					(w1)	[right of=0] 	{$q_{w_1}$}; 
   \node[state]						(ab) 	[right of=w1] 	{$q_{ab}$};
   \node[state,accepting]			(w2) 	[right of=ab] 	{$q_{w_2}$};
   
   \path[->]
    (0) edge [bend left] node {$ w_1 $} (w1)  
    (w1) edge [bend left] node {a} (ab)
    (ab) edge [bend left] node {$ w_2 $} (w2)
    (ab) edge [loop above] node {b} ()
    ;
\end{tikzpicture}
\caption{Rozklad automatu $ A_w $ na automaty $ A_w^a $ (hore) a $ A_w^b $ (dole)}
\end{figure}

Ľahko vidno, že $ L(A_w^a) = \lbrace w_1a^kbw_2 | k \in \mathbb{N} \rbrace $ a $ L(A_w^b) = \lbrace w_1ab^kw_2 | k \in \mathbb{N} \rbrace$. Ukážeme, že $ L(A_w^a) \cap L(A_w^b) = \lbrace w \rbrace $. \\ 
$ \subseteq: $ Nech $ u \in L(A_w^a) \cap L(A_w^b)  $. Teda existujú $ k_1,k_2 \in \mathbb{N} $ také, že $ u = w_1a^{k_1}bw_2 = w_1ab^{k_2}w_2 $. Musí platiť $ k_1 = k_2 $, inak by platilo $ |u| \neq |u| $. Taktiež musí platiť $ k_1 \geq 1 $, lebo $ pref(u,|w_1|+1) = w_1a $. Teda aj $ k_2 \geq 1 $, lebo inak by bolo $ k_1 \neq k_2 $. Teda $ pref(u,|w_1|+2) = w_1ab $. Z toho nutne $ k_1 = 1 $ a teda aj $ k_2 = 1 $. Takže $ u = w_1abw_2 = w $. \\
$ \supseteq: $ Táto inklúzia je zjavná.
\par
Navyše platí $ \#_S(A_w^a) < \#_S(A_w) $ a $ \#_S(A_w^b) < \#_S(A_w) $. Takže $ A_w^a $ a $ A_w^b $ tvoria netriviálny rozklad automatu $ A_w $, čo ukazuje, že jazyk $ L $ je rozložiteľný.

\end{proof}















