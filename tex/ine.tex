\chapter[Iné výsledky]{Iné výsledky}
\label{kap:properties_of_classes}

Uvidíme čo kde ešte dať tak zatial sem

\section{Porovnanie deterministickej a nedeterministickej rozložiteľnosti regulárnych jazykov}

Zaujímavou otázkou je, či existuje regulárny jazyk taký, že je deterministicky nerozložiteľný a súčasne nedeterminiticky rozložiteľný respektíve deterministicky rozložiteľný a súčasne nedeterminiticky nerozložiteľný. Pred tým, ako uvedieme dosiahnuté výsledky zavedieme definícu deterministického konečného automatu, ktorú budeme používať, nakoľko existuje viacero prístupov k definovaniu deterministických konečných automatov.

\begin{definition}
\textbf{Deterministický konečný automat} je pätica $ (K, \Sigma, \delta, q_0, F) $, kde:
\begin{enumerate}  
\item $ K $ je konečná množina stavov
\item $ \Sigma $ je konečná vstupná abeceda
\item $ q_0 \in K $ je počiatočný stav
\item $ F \subseteq K $ je množina akceptačných stavov
\item $ \delta : K \times \Sigma \rightarrow K $ je prechodová funkcia
\end{enumerate}
\end{definition}

\begin{note}
Deterministický konečný automat sa skrátene označuje DKA.
\end{note}

Poznajúc ako v našom texte definujeme deterministický konečný automat je pre čitateľa so základnými znalosťami v oblasti jasné, ako by boli definované ostatné potrebné pojmy, preto ich definície neuvádzame.

\begin{theorem}
Existuje nedeterministicky nerozložiteľný deterministicky rozložiteľný regulárny jazyk.
\end{theorem}

\begin{proof}
Hľadaným jazykom je jazyk $ L = (\lbrace a \rbrace \lbrace a,b \rbrace \lbrace a \rbrace \lbrace a,b \rbrace)^* $. Ukážeme, že jazyk $ L $ je deterministicky rozložiteľný. Najprv zostrojíme minimálny DKA $ A_{L} $ akceptujúci $ L $. Automat uvádzame pomocou diagramu.

\begin{figure}[H]
\centering
\begin{tikzpicture}[->,>=stealth',shorten >=1pt,auto,node distance=2.5cm,
                    semithick]
   \node[state,initial,accepting] 	(0) 					{$q_0$}; 
   \node[state] 					(1)		[right of=0] 	{$q_1$}; 
   \node[state]						(2)		[right of=1]	{$q_2$};
   \node[state]						(3) 	[right of=2] 	{$q_3$};
   \node[state]						(T) 	[below of=1] 	{$q_T$};
   
   \path[->]
    (0) edge [bend right] node {a} 	(1)  
    (1) edge [bend right] node {a,b} (2)
    (2) edge [bend right] node {a} (3)
    (3) edge [bend right] node [swap] {a,b} (0)
    (0) edge node {b} (T)
    (2) edge node {b} (T)
    (T) edge [loop right] node {a,b} ()
    ;
\end{tikzpicture}
\caption{deterministický konečný automat $ A_{L} $ pre jazyk $ L = (\lbrace a \rbrace \lbrace a,b \rbrace \lbrace a \rbrace \lbrace a,b \rbrace)^* $} \label{fig:DFAa(a|b)a(a|b)^4}
\end{figure}

Ľahko vidno, že $ A_L $ akceptuje práve $ L $. Minimalita $ A_L $ sa dá dokázať pomocou všeobecne známej Myhill-Nerodeovej vety. Zostrojíme netriviálny rozklad automatu $ A_L $. Hľadané DKA $ A_1^L $ a $ A_2^L $ uvádzame pomocou ich diagramov.

\begin{figure}[H]
\centering
\begin{tikzpicture}[->,>=stealth',shorten >=1pt,auto,node distance=2.5cm,
                    semithick]
   \node[state,initial,accepting] 	(0) 					{$q_0$}; 
   \node[state] 					(1)		[right of=0] 	{$q_1$}; 
   \node[state]						(2)		[right of=1]	{$q_2$};
   \node[state]						(3) 	[right of=2] 	{$q_3$};
   
   \path[->]
    (0) edge [bend right] node {a,b} (1)  
    (1) edge [bend right] node {a,b} (2)
    (2) edge [bend right] node {a,b} (3)
    (3) edge [bend right] node [swap] {a,b} (0)
    ;
\end{tikzpicture}
\begin{tikzpicture}[->,>=stealth',shorten >=1pt,auto,node distance=2.5cm,
                    semithick]
   \node[state,initial,accepting] 	(0) 					{$q_0$}; 
   \node[state] 					(1)		[right of=0] 	{$q_1$}; 
   \node[state]						(T) 	[below right of=0] 	{$q_T$};
   
   \path[->]
    (0) edge [bend right] node {a} 	(1)  
    (1) edge [bend right] node [swap] {a,b} (0)
    (0) edge node [swap] {b} (T)
    (T) edge [loop right] node {a,b} ()
    ;
\end{tikzpicture}
\caption{rozklad automatu $ A_L $}
\end{figure}

Možno nahliadnuť, že jeden z automatov v rozklade počíta zvyšok po delení 4 a druhý kontroluje, či symboly na nepárnych pozíciách v slove sú $ a $. Teda vidno, že $ L(A_1^L) = \lbrace w \in \lbrace a,b \rbrace^* \; | \; |w| \equiv 0 \; (mod \; 4) \rbrace $ a $ L(A_2^L) = (\lbrace a \rbrace \lbrace a,b \rbrace)^* $. Teda $ L(A_1^L) \cap L(A_2^L) = L $. Navyše $ \#_S(A_1^L) < \#_S(A_L) $ a $ \#_S(A_2^L) < \#_S(A_L) $, teda automaty $ A_1^L $ a $ A_2^L $ tvoria netriviálny rozklad automatu $ A_L $. Z predchádzajúceho vyplýva, že jazyk $ L = (\lbrace a \rbrace \lbrace a,b \rbrace \lbrace a \rbrace \lbrace a,b \rbrace)^* $ je deterministicky rozložiteľný. Avšak tento jazyk je podľa Vety \ref{thm:a(a|b)a(a|b)^4} nedeterministicky nerozložiteľný.
\end{proof}

\section{Automaty tvorené jediným cyklom}

Typickou schopnosťou konečných automatov je počítať v cykle zvyšok po delení dĺžky slova. Tieto automaty sa vyznačujú tým, že sú tvorené jediným cyklom, pričom nijak nezohľadňujú štruktúru slova. Podstatu otázok spojených s takýmito automatmi riešia Vety \ref{thm:prime^n} a \ref{thm:compound}. Nakoľko v konečných automatoch sú práve cykly veľmi dôležitou štruktúrou, v našej práci sme túto otázku rozšírili a študovali sme otázku rozložiteľnosti jazykov, ktorých minimálne nedeterministické konečné automaty sú tvorené jediným cyklom, pričom v ňom zohľadňujú aj štruktrúru akceptovaného slova. Podstatou týchto automatov je, neformálne povedané, pumpovanie nejakého slova.
\par
Pre lepšiu čitateľnosť dôkazov zavedieme nasledujúce označenia.

\begin{notation}
\normalfont
Nech $ u $ je ľubovolné slovo, $ k \in \mathbb{N} $. Potom $ pref(u,k) $ označujeme prefix slova $ u $ dĺžky $ k $ a $ suff(u,k) $ označujeme suffix slova $ u $ dĺžky $ k $.
\end{notation}

\begin{notation}
\normalfont
Nech $ u = u_1u_2 \ldots u_n$ je ľubovolné slovo. Ak v diagrame NKA $ A $ použijeme nasledujúce označenie:
\begin{figure}[H]
\centering
\begin{tikzpicture}[->,>=stealth',shorten >=1pt,auto,node distance=2.5cm,
                    semithick]
   \node[state] 	(1)		 				{$q$}; 
   \node[state]		(2)		[right of=1]	{$p$};
   
   \path[->]
    (1) edge node {u} (2)  
    ;
\end{tikzpicture}
\end{figure}

Myslíme tým, že v automate $ A $ sa dá zo stavu $ q $ dostať do stavu $ p $ na slovo $ u $ pričom zo stavov, v ktorých sa automat $ A $ nachádza počas čítania slova $ u $ sa nedá už nikam inam dostať. Formálne existujú $ q_0, q_1, \ldots, q_n \in K_A $ také, že $ q_0 = q, q_n = p, \delta_A(q, u_1) \ni q_1 $ a pre $ 0 < i < n $ platí $ \delta_A(q_i,u_{i+1}) = \lbrace q_{i+1} \rbrace, q_i \notin F_A $. Treba si uvedomiť, že pokiaľ $ u = \varepsilon $, tak platí $ q=p $, ak navyše v tom prípade aspoň jeden zo stavov je označený v diagrame ako akceptačný, tak sa tým myslí, že stav je akceptačný.
\end{notation}

\begin{lemma}
\label{lm:one_word_cycle_nsc}
Nech $ \Sigma $ je ľubovolná abeceda, nech $ u \in \Sigma^* $, nech $ L_u = \lbrace u \rbrace^* $. Potom $ nsc(L_u) = |u| $.
\end{lemma}

\begin{proof}
Zostrojíme NKA $ A_u $ pre jazyk $ L_u $. Automat uvádzame pomocou diagramu.

\begin{figure}[H]
\centering
\begin{tikzpicture}[->,>=stealth',shorten >=1pt,auto,node distance=2cm,
                    semithick]
   \node[state,initial,accepting] 	(0) 					{$q_0$}; 
   
   \path[->]
    (0) edge [loop right] node {u} ()  
    ;
\end{tikzpicture}
\caption{automat $ A_u $}
\end{figure}
Ľahko vidno, že $ L(A_u) = L_u $. Uvažujme množinu dvojíc slov $ F = \lbrace (pref(u,i), \; suff(|u|-i)) \; | \; 0 \leq i < |u| \rbrace $. Množina $ F $ je podľa definície \ref{def:fooling_set} oblbovacou množinou pre jazyk $ L_u $. Nakoľko $ |F|=|u| $, tak podľa Vety \ref{thm:fooling_set_technique} $ nsc(L_u) \geq |u| $. Kedže $ L(A_u) = L_u $ a $ \#_S(A_u) = |u| $, tak $ nsc(L_u) = |u| $ a automat $ A_u $ je minimálny NKA pre jazyk $ L_u $.
\end{proof}

\begin{theorem}
Nech $ \Sigma $ je ľubovolná abeceda taká, že $ |\Sigma| \geq 2 $. Nech pre $ u \in \Sigma^* $, $ k \geq 2 $ je $ L_u^k = \lbrace u^k \rbrace^* $. Ak $ u $ obsahuje aspoň dva rôzne symboly, potom je $ L_u^k $ rozložiteľný.
\end{theorem}

\begin{proof}
Nech $ n \geq 1, \Sigma = \lbrace a, b_1, \ldots , b_n \rbrace, u \in \Sigma^*,$ $ u $ obsahuje symbol $ a $ a minimálne ešte jeden symbol zo $ \Sigma $. Podľa Lemy \ref{lm:one_word_cycle_nsc} platí $ nsc(L_u^k) = k|u| $. Teda existuje NKA $ A_u^k $ taký, že $ L(A_u^k) = L_u^k $ a $ \#_S(A_u^k) = k|u| $. Automat $ A_u^k $ je teda minimálny NKA pre $ L_u^k $. Zostrojíme netriviálny rozklad automatu $ A_u^k $. Označme $ l = k.\#_a(u) $. Rozklad uvádzame pomocou diagramu.

\begin{figure}[H]
\centering
\begin{tikzpicture}[->,>=stealth',shorten >=1pt,auto,node distance=2cm,
                    semithick]
   \node[state,initial,accepting] 	(0) 					{$q_0$}; 
   
   \path[->]
    (0) edge [loop above] node [swap] {u} ()  
    ;
\end{tikzpicture}

\begin{tikzpicture}[->,>=stealth',shorten >=1pt,auto,node distance=2.21cm,
                    semithick]
   \node[state,initial] 	(0) 					{$[0]$}; 
   \node[state] 			(1)		[right of=0] 	{$[1]$}; 
   \node[state] 			(2)		[right of=1] 	{$[2]$};
   \node					(dots)	[right of=2]	{$\ldots$};
   \node[state]				(auk_1) 	[right of=dots] 	{$[l-1]$};
   \node[state,accepting]	(auk) 	[right of=auk_1] 	{$[l]$};
   
   \path[->]
    (0) edge [bend left] node {a} 	(1)
    (1) edge [bend left] node {a} 	(2)
    (2) edge [bend left] node {a} 	(dots)
    (dots) edge [bend left] node {a} 	(auk_1)
    (auk_1) edge [bend left] node {a} 	(auk)
    (auk) edge [bend left = 35] node {a} 	(0)
   	(0) edge [loop above] node {$ b_1, \ldots, b_n $} ()
   	(1) edge [loop above] node {$ b_1, \ldots, b_n $} ()
   	(2) edge [loop above] node {$ b_1, \ldots, b_n $} ()
   	(auk_1) edge [loop above] node {$ b_1, \ldots, b_n $} ()
 	(auk) edge [loop above] node {$ b_1, \ldots, b_n $} ()
    ;
\end{tikzpicture}

\caption{rozklad automatu $ A_u^k $ na automaty $ A_u $(hore) a $ A_k $(dole)}
\end{figure}

Myšlienkou tohto rozkladu je, že jeden z automatov kontroluje štruktúru slova, či je práve niekoľkonásobným zreťazením slova $ u $ a druhý automat kontroluje, či je slov $ u $ správne veľa. To však robí tak, že iba počíta počet nejakého jedného symbolu (v našom prípade ho označujeme $ a $), ktorý $ u $ obsahuje, pričom kontroluje, či slovo obsahuje práve $ m.k.\#_a(u) $ pre nejaké $ m \in \mathbb{N} $. Formálne $ L(A_u) = \lbrace u \rbrace^* $ a $ L(A_k) = \lbrace w \in \Sigma^* \; | \; \#_a(w) \equiv 0 \; (mod \; k.\#_a(u)) \rbrace $. Teda $ L(A_u) \cap L(A_k) = L(A_u^k) $. Navyše $ \#_S(A_u) < \#_S(A_u^k) $ a $ \#_S(A_k) < \#_S(A_u^k) $. Je dobré si uvedomiť, že kvôli prvej nerovnosti potrebujeme predpoklad $ k \geq 2 $ a kvôli druhej nerovnosti potrebujeme predpoklad o veľkosti abecedy $ \Sigma $. Teda automaty $ A_u $ a $ A_k $ tvoria netriviálny rozklad automatu $ A_u^k $.

\end{proof}

\begin{theorem}
Nech $ \Sigma $ je ľubovolná abeceda, nech $ k_1,k_2 \in \lbrace 0,1 \rbrace $, nech $ w_1,w_2,w_3,w_4,w_5,w_6 \in \Sigma^* $. Definujeme $ L = \lbrace w_1a^{k_1}w_2bw_3aw_4bw_5a^{k_2}w_6 \rbrace^* $. Ak $ k_1 = 1 $ alebo $ k_2 = 1 $, potom je $ L $ rozložiteľný.
\end{theorem}

\begin{proof}
Zaveďme označenia $ u = w_1a^{k_1}w_2bw_3aw_4bw_5a^{k_2}w_6$ a $ \Sigma_{ab} = \Sigma \cup \lbrace a,b \rbrace $. Podľa Lemy \ref{lm:one_word_cycle_nsc} platí $ nsc(L) = |u| $. Teda existuje NKA $ A $ taký, že $ L(A) = L $ s $ \#_S(A) = |u| $. Automat $ A $ je teda minimálny NKA pre $ L $. Zostrojíme netriviálny rozklad automatu $ A $. Rozoberieme nasledujúce dva prípady, podľa toho akého tvaru je slovo $ u $. Podľa predpokladov je $ u $ práve jedého z nasledujúcich tvarov:

\begin{enumerate}
\item Existujú dve rôzne podslová v slove $ u $ také, že symbol $ b $ je nasledovaný symbolom rôznym od $ b $. Formálne existujú $ v_1,v_2,v_3 \in \Sigma_{ab}^* $ a $ \overline{b}_1,\overline{b}_2 \in \Sigma_{ab} - \lbrace b \rbrace $ také, že $ u = v_1b\overline{b}_1v_2\overline{b}_2v_3 $. Na základe tohto poznatku zostrojíme netriviálny rozklad automatu $ A $. Rozklad uvádzame pomocou diagramu.

\begin{figure}[H]
\centering
\begin{tikzpicture}[->,>=stealth',shorten >=1pt,auto,node distance=2cm,
                    semithick]
	\node[state,initial,accepting] (0)							{$q_0$};
   	\node[state] 					(v1)	[right of=0]		{$q_{v_1}^{b}$};
   	\node[state] 					(notb1)	[right of=v1]		{$q_{\overline{b}_1}$};      	\node[state] 					(v2)	[right of=notb1]	{$q_{v_2}$};
   	\node[state] 					(b)		[right of=v2]		{$q_b$};
   	\node[state] 					(notb2)	[right of=b]		{$q_{\overline{b}_2}$};
   
   	\path[->]
     (0) edge [bend left] node {$ v_1 $} (v1) 
     (v1) edge [bend left] node {$ \overline{b}_1 $} (notb1)
     (v1) edge [loop above] node {$ b $} ()
     (notb1) edge [bend left] node {$ v_2 $} (v2)
     (v2) edge [bend left] node {$ b $} (b)
     (b) edge [bend left] node {$ \overline{b}_2 $} (notb2)
     (notb2) edge [bend left] node {$ v_3 $} (0)
     ;
\end{tikzpicture}

\begin{tikzpicture}[->,>=stealth',shorten >=1pt,auto,node distance=2cm,
                    semithick]
	\node[state,initial,accepting] (0)							{$q_0$};
   	\node[state] 					(v1)	[right of=0]		{$q_{v_1}$};
 	\node[state] 					(b)		[right of=v1]		{$q_b$};
   	\node[state] 					(notb1)	[right of=b]		{$q_{\overline{b}_1}$}; 	
	\node[state] 					(v2)	[right of=notb1]	{$q_{v_2}^b$};
   	\node[state] 					(notb2)	[right of=v2]		{$q_{\overline{b}_2}$};
   
   	\path[->]
     (0) edge [bend left] node {$ v_1 $} (v1) 
     (v1) edge [bend left] node {$ b $} (b)
     (b) edge [bend left] node {$ \overline{b}_1 $} (notb1)
     (notb1) edge [bend left] node {$ v_2 $} (v2)
     (v2) edge [loop above] node {$ b $} ()
     (v2) edge [bend left] node {$ \overline{b}_2 $} (notb2)
     (notb2) edge [bend left] node {$ v_3 $} (0)
     ;
\end{tikzpicture}
\caption{rozklad automatu $ A $ na automaty $ A_1 $ a $ A_2 $}
\end{figure}

Možno nahliadnuť, že $ L(A_1) = \lbrace v_1b^{l}\overline{b}_1v_2b\overline{b}_2v_3 \; | \; l \in \mathbb{N} \rbrace^* $ a $ L(A_2) = \lbrace v_1b\overline{b}_1v_2b^{l}\overline{b}_2v_3 \; | \; l \in \mathbb{N} \rbrace^* $.
\par 
Dokážeme, že $ L(A_1) \cap L(A_2) = L$. \\
$ \supseteq: $ Táto inklúzia je triviálna, nebudeme ju formálne dokazovať. \\
$ \subseteq: $ Uvažujme $ w \in L(A_1) \cap L(A_2) $. Potom existuje $ n,m,l_1,\ldots,l_n,o_1,\ldots,o_m \in \mathbb{N} $ také, že $ w = v_1b^{l_1}\overline{b}_1v_2b\overline{b}_2v_3 \ldots v_1b^{l_n}\overline{b}_1v_2b\overline{b}_2v_3 = v_1b\overline{b}_1v_2b^{o_1}\overline{b}_2v_3 \ldots v_1b\overline{b}_1v_2b^{o_m}\overline{b}_2v_3 $. Indukciou na $ n $ dokážeme, že $ m=n $, pre $ 0 \leq i \leq n: l_i=1 $ a pre $ 0 \leq i \leq m: o_i=1 $.

\begin{itemize}
    \item [$ 1^0: $] Ak $ n=0 $, tak $ w = \varepsilon $ a tvrdenie triviálne platí.
    \item [$ 2^0: $] Platí $ v_1b^{l_1}\overline{b}_1v_2b\overline{b}_2v_3 \ldots v_1b^{l_n}\overline{b}_1v_2b\overline{b}_2v_3 = v_1b\overline{b}_1v_2b^{o_1}\overline{b}_2v_3 \ldots v_1b\overline{b}_1v_2b^{o_m}\overline{b}_2v_3 $. Pozrime sa pozornejšie na prvé úseky v tomto slove, t.j. na časti $ v_1b^{l_1}\overline{b}_1v_2b\overline{b}_2v_3 $ a $ v_1b\overline{b}_1v_2b^{o_1}\overline{b}_2v_3 $. Oba úseky sú prefixom toho istého slova a na prvých $ |v_1| $ symboloch sa evidentne zhodujú. Musí platiť $ l_1 \geq 1 $, aby sa zhodovali aj na symoble $ b $, ktorý nasleduje za $ v_1 $. Avšak nakoľko v tomto prefixe po zmienenom $ b $ nasleduje znak $ \overline{b}_1 $, tak nutne $ l_1=1 $. Teda platí $ v_1b^{l_1}\overline{b}_1v_2 = v_1b\overline{b}_1v_2 $. Z toho plynie $ o_1 \geq 1 $, nakoľko po $ v_2 $ musí nasledovať symbol $ b $. Ďalším sybmobolom je však $ \overline{b}_2 $, teda nutne $ o_1 = 1 $. Teda platí $ v_1b^{l_1}\overline{b}_1v_2b\overline{b}_2v_3 = v_1b\overline{b}_1v_2b^{o_1}\overline{b}_2v_3 = v_1b\overline{b}_1v_2b\overline{b}_2v_3$. Navyše, oba automaty, $ A_1 $ aj $ A_2 $ sa po dočítaní tohto prefixu dostanú práve do ich počiatočného (a zároveň jediného akceptačného) stavu $ q_0 $. V prípade, že $ n=1 $, tak niet čo ďalej dokazovať. Ak $ n \geq 2 $ tak z predchádzajúceho vyplýva, že $ v_1b^{l_2}\overline{b}_1v_2b\overline{b}_2v_3 \ldots v_1b^{l_n}\overline{b}_1v_2b\overline{b}_2v_3 = v_1b\overline{b}_1v_2b^{o_2}\overline{b}_2v_3 \ldots v_1b\overline{b}_1v_2b^{o_m}\overline{b}_2v_3 $ a navyše toto slovo akceptujú oba automaty, $ A_1 $ aj $ A_2 $. Teda podľa indukčného predpokladu môžeme tvrdiť, že $ n=m $, pre $ 2 \leq i \leq n $ platí $ l_i = o_i = 1 $, čo dokazuje tvrdenie.
\end{itemize}
Z predošlého vyplýva, že $ w \in L $, čo kompletizuje dôkaz tejto inklúzie. 
\par
Teda $ L(A_1) \cap L(A_2) = L = L(A) $. Navyše $ \#_S(A_1) < \#_S(A) $ a $ \#_S(A_2) < \#_S(A) $, teda automaty $ A_1 $ a $ A_2 $ tvoria netriviálny rozklad automatu $ A $.

\item Existujú $ \overline{b}_1,\overline{b}_2 \in \Sigma_{ab} - \lbrace b \rbrace $, $ v_1,v_2 \in \Sigma_{ab} - \lbrace b \rbrace $, $ c_1,c_2 \geq 1 $ také, že $ u = \overline{b}_1v_1b^{c_1}\overline{b}_2v_2b^{c_2} $. Na základe tohto poznatku zostrojíme netriviálny rozklad automatu $ A $. Rozklad uvádzame pomocou diagramu.

\begin{figure}[H]
\centering
\begin{tikzpicture}[->,>=stealth',shorten >=1pt,auto,node distance=2cm,
                    semithick]
	\node[state,initial,accepting] (0)							{$q_0$};
   	\node[state] 					(notb1)		[right of=0]	{$q_{\overline{b}_1}$};
   	\node[state] 					(v1)		[right of=notb1]	{$q_{v_1}^b$};
   	\node[state] 					(notb2)		[right of=v1]	{$q_{\overline{b}_2}$};
   	\node[state] 					(v2)		[right of=notb2] {$q_{v_2}$};
   
   	\path[->]
     (0) edge [bend left] node {$ \overline{b}_1 $} (notb1)
     (notb1) edge [bend left] node {$ v_1 $} (v1)
     (v1) edge [loop above] node {$ b $} ()
     (v1) edge [bend left] node {$ \overline{b}_2 $} (notb2)
     (notb2) edge [bend left] node {$ v_2 $} (v2)
     (v2) edge [bend left] node {$ b $} (0)
     ;
\end{tikzpicture}
\begin{tikzpicture}[->,>=stealth',shorten >=1pt,auto,node distance=2cm,
                    semithick]
	\node[state,initial,accepting] (0)							{$q_0^b$};
   	\node[state] 					(notb1)		[right of=0]	{$q_{\overline{b}_1}$};
   	\node[state] 					(v1)		[right of=notb1]	{$q_{v_1}$};
   	\node[state] 					(b)			[right of=v1]	{$q_{b}$};
   	\node[state] 					(notb2)		[right of=b]	{$q_{\overline{b}_2}$};
   
   	\path[->]
     (0) edge [bend left] node {$ \overline{b}_1 $} (notb1)
     (notb1) edge [bend left] node {$ v_1 $} (v1)
     (v1) edge [bend left] node {$ b $} (b)
     (b) edge [bend left] node {$ \overline{b}_2 $} (notb2)
     (notb2) edge [bend left] node {$ v_2 $} (0)
     (0) edge [loop above] node {$ b $} ()
     ;
\end{tikzpicture}
\caption{rozklad automatu $ A $ na automaty $ A_1 $(hore) a $ A_2 $(dole)}
\end{figure}

Možno nahliadnuť, že $ L(A_1) = \lbrace \overline{b}_1v_1b^l\overline{b}_2v_2b \; | \; l \in \mathbb{N} \rbrace^* $ a $ L(A_2) = \lbrace b^l\overline{b}_1v_1b\overline{b}_2v_2 \; | \; l \in \mathbb{N} \rbrace^*\lbrace b \rbrace^* $. Platí $ L(A_1) \cap L(A_2) = L $, čo sa dá dokázať veľmi podobne a rovnako veľmi technicky ako v predošlom prípade, preto dôkaz neuvádzame. Navyše $ \#_S(A_1) < \#_S(A) $ a $ \#_S(A_2) < \#_S(A) $, teda automaty $ A_1 $ a $ A_2 $ tvoria netriviálny rozklad automatu $ A $.
\end{enumerate}

Záverom ešte spomeňme, že hlavnou myšlienkou rozkladu bola akási synchronizácia výpočtov automatov v rozklade na symboloch rôznych od $ b $, ktoré nasledovali hneď za $ b $.
\end{proof}
















